\section{The Calculus $\Calcw$}
\label{sec_calculus}

%%#############################  CALCULUS - PROPOSITIONAL RULES  ################################
\begin{figure}[t]
  \[%\large
  \centering
  \begin{array}{c}
    %% AXAT 
    \mbox{$\lt\in\{\,<,\,\leq\,\}$,\hspace{1em} $\gt \in \{\,>,\,\geq\,\}$}\qquad\qquad
    \AXC{}
    \RightLabel{\small$\ruleAxat$}
    \UIC{$\G$}    
    \DP
    \quad\mbox{if $\Atmp{\G}$ is not consistent}
    \\[4ex]
    %% AND GT 
    \AXC{$w: \a \gt t,\, w: \b\gt t,\,  \G$}    
    % ------------------------------------------    
    \RightLabel{$\land\gt$}
    \UIC{$w: \a\land \b \gt t,\,\G$}
    \DP
    \hspace{4em}
    %% AND LT 
    \AXC{$w: \a \lt t,\, \G$}    
    \AXC{$w: \b \lt t,\, \G$}
    % ------------------------------------------      
    \RightLabel{$\land\lt$}
    \BIC{$w: \a\land \b \lt t,\,\G$}
    \DP
    \\[4ex]    
    %%%%%%%%%%%%%%%%%%%%%%%%%%%%%%%%%%%%%%%%%%%%%%%%%%%%%%%% 
    %% OR LT 
    \AXC{$w: \a \lt t,\, w: \b \lt t,\,  \G$}    
    % ------------------------------------------    
    \RightLabel{$\lor\lt$}
    \UIC{$w: \a\lor \b \lt t,\,\G$}
    \DP
    \hspace{4em}
    %% OR GT 
    \AXC{$w: \a \gt t,\,  \G$}
    \AXC{$w: \b \gt t,\,  \G$}
    % ------------------------------------------        
    \RightLabel{$\lor\gt$}
    \BIC{$w: \a\lor \b \gt t,\,\G$}
    \DP
    \\[4ex]
    %%%%%%%%%%%%%%%%%%%%%%%%%%%%%%%%%%%%%%%%%%%%%%%%%%%%%%%%%%%%%%%%%%%%%% 
    %% IMP LTS prop
    \AXC{$w :\a >w : p,\, w : p  < t,\,  \G$}    
    % ------------------------------------------    
    \RightLabel{$\ruleToLTAt\;(\dag)$}
    \UIC{$w: \a\to p < t,\,\G$}
    \DP
    \hspace{1em}
    %% IMP LTS 
    \AXC{$w :\a > w : q,\,  w :\b \leq w :q,\,  w : q  < t,\,  \G$}    
    % ------------------------------------------    
    \RightLabel{$\ruleToLT\;(\dag)$}
    \UIC{$w: \a\to \b < t,\,\G$}
    \DP
    \\[4ex]
    %% IMP LTE prop
    \AXC{$t \geq 1,\, \G$}    
    \AXC{$w :\a >w : p,\,w : p \leq  t,\,\G$}    
    % ------------------------------------------    
   % \insertBetweenHyps{\hskip -0.01em}
    \RightLabel{$\ruleToLEQAt\,(\dag)$}
    \BIC{$w: \a\to p \leq t,\,\G$}
    \DP
\\[4ex]    
    %% IMP LTE
    \AXC{$t \geq 1,\, \G$}    
    \AXC{$  w :\a > w : q,\,    w :\b \leq w : q,\,   w : q \leq  t,\,\G$}    
    % ------------------------------------------    
    \RightLabel{$\ruleToLEQ\,(\dag)$}
    \BIC{$w: \a\to \b \leq t,\,\G$}
    \DP
    \\[4ex]   
 %% IMP GT prop
    \AXC{$w :\a \leq w : p,\,1\gt t,\, \G$}    
    \AXC{$w:p \gt t,\, \G$}
    % ------------------------------------------       
    \RightLabel{$\ruleToGAt\,(\dag)$}
    \BIC{$w: \a\to p \gt t,\,\G$}
    \DP
    \\[4ex]
    %% IMP GT 
    \AXC{$w :\a\ \leq w : q,\, w : \b \geq w :q ,\, 1\gt t,\, \G$}    
    \AXC{$w:\b \gt t,\, \G$}
    % ------------------------------------------       
    \RightLabel{$\ruleToG\,(\dag)$}
    \BIC{$w: \a\to \b \gt t,\,\G$}
    \DP
    %%% WHERE ....
    \\[4ex]
    \begin{minipage}{1.0\linewidth}
      \[
      \begin{array}{l}
        \Atmp{\G} \;=\;
        \Atm{\G}\;\cup\;
        \{\, 1 > t~|~ w : \Box \a > t \in \G\,\}\;\cup\;
        \{\, 0 < t~|~ w : \Diam \a < t \in \G\,\}
        \\[1ex]   
        (\dag)\quad p\in\PV\cup\{\bot\},\;  \b\not\in\PV\cup\{\bot\},\;\mbox{$q$ is a new propositional variable}
        \end{array}
       \]
     \end{minipage}   
  \end{array}
  \]
  \vspace{-2ex}
 \caption{The calculus $\Calcw$, propositional rules ($\Gamma$ is a multiset of constraints).}
  \label{fig:calc1}
\end{figure}
%################



%%#############################  CALCULUS - MODAL RULES  AND R-RULES ################################
\begin{figure}[t]
  \[%\large
  \centering
  \begin{array}{c}
    %% BOX LT        
    \AXC{$R(w,w_1)\to w_1:\a \lt t,\;\Phibd{\G,w,w_1},\;\G$}
    % -----------------------------------------
    \RightLabel{$\Box\lt$}
    \UIC{$w:\Box\a \lt t,\, \G$}    
    \DP
    \hspace{3em}
    %% DIAM GT L     
    \AXC{$R(w,w_1)\land w_1:\a \gt t,\;\Phibd{\G,w,w_1},\;  \G$}
    % ------------------------------------------
    \RightLabel{$\Diam\gt$}
    \UIC{$w:\Diam\a \gt t,\,\G$}    
    \DP
    %%% WHERE ....
    \\[2ex]
     \begin{minipage}{1.0\linewidth}
       \[
       \begin{array}{l}
         \mbox{$w_1$ is a new label}
         \\[1ex]
         \Phibd{\G,w, w_1}\,=\,
         \{\, R(w,w_1)\to w_1: \b \gt t~|~w :\Box \b\gt t \in\G  \,\}\,\cup\,
         \{\, R(w,w_1)\land w_1: \b \lt t~|~w :\Diam \b\lt t \in\G  \,\}
       \end{array}
       \]
     \end{minipage}
     \\[6ex]
    %% R AND GT 
    \AXC{$R(w,w') \gt t,\, w': \a\gt t,\,  \G$}    
    % ------------------------------------------    
    \RightLabel{$R\land\gt$}
    \UIC{$R(w,w') \land w': \a \gt t,\,\G$}
    \DP
    \hspace{2em}
    %% R AND LT 
    \AXC{$R(w,w') \lt t,\, \G$}    
    \AXC{$w': \a \lt t,\, \G$}
    % ------------------------------------------      
    \RightLabel{$R\land\lt$}
    \BIC{$R(w,w') \land w':\a \lt t,\,\G$}
    \DP
    \\[4ex]
    % ==============================================================
    %% R IMP LTS 
    \AXC{$w':\a<R(w,w'),\,  w':\a  < t,\,  \G$}    
    % ------------------------------------------    
    \RightLabel{$R\to <$}
    \UIC{$R(w,w') \to w':\a < t,\,\G$}
    \DP
    \hspace{2em}
    %%\\[4ex]
    %% %% R IMP LTS 
    %% \AXC{$R(w,w')> w': q,\,w':\a \leq w':q,\,  w':q  < t,\,  \G$}    
    %% % ------------------------------------------    
    %% \RightLabel{$R\to <\,(\dag)$}
    %% \UIC{$R(w,w') \to w':\a < t,\,\G$}
    %% \DP
    %% \\[4ex]    
    %% R IMP LTE 
    \AXC{$t \geq 1,\, \G$}    
    \AXC{$w':\a<R(w,w'),\,  w':\a \leq  t,\,\G$}    
    % ------------------------------------------    
    \RightLabel{$R\to\leq$}
    \BIC{$R(w,w')\to w':\a \leq t,\,\G$}
    \DP
    \\[4ex]    
    %% %% R IMP LTE 
    %% \AXC{$t \geq 1,\, \G$}    
    %% \AXC{$R(w,w')> w':q,\,w':\a\leq w':q,\,  w':q \leq  t,\,\G$}    
    %% % ------------------------------------------    
    %% \RightLabel{$R\to\leq\,(\dag)$}
    %% \BIC{$R(w,w')\to w':\a \leq t,\,\G$}
    %% \DP
    %% \\[4ex]   
    %% R IMP GT 
    \AXC{$w' :\a \geq R(w,w'),\,1\gt t,\, \G$}    
    \AXC{$w':\a \gt t,\, \G$}
    % ------------------------------------------       
    \RightLabel{$R\to\gt$}
    \BIC{$R(w,w')\to w':\a \gt t,\,\G$}
    \DP
    %%% WHERE ....
    %% (\dag)\quad p\in\PV\cup\{\bot\},\;  \a\not\in\PV\cup\{\bot\},\;\mbox{$q$ is a new propositional variable}
  \end{array}
  \]
  \vspace{-2ex}
  \caption{The calculus $\Calcw$: modal rules and $R$-rules ($\Gamma$ is a multiset of constraints).}
  \label{fig:calc2}
\end{figure}
%#########################################################################



Following~\cite{BilkovaFK:22,FerFioRod:2025}, we introduce the
refutation calculus $\Calcw$ for $\GWL$, namely, a calculus to certify
that a formula is not valid in any $\GWM$-model.  The calculus
exploits the constraint language $\LC$ defined over a countable set
of labels, each representing a world of a \GWM-model.  In the
following definitions $w$, $w'$ are labels of $\LC$, $\varphi$ is a
formula, $p\in \PV$, $R$ is a designated relation symbol, representing
the accessibility relation.
%and $\abstractorder\in \{<,\leq,>,\geq\,\}$.
\[
\begin{array}{rcl}
  \mbox{labelled formula} & \;\coloneqq&\; w :\varphi                            
                             \\
  \mbox{atomic c-term $t$}  &\; \coloneqq&\; 0~|~1~|~w :p~|~w :\bot~|~R(w,w')
  \\[.5ex]
  \mbox{c-term $u$}  & \;\coloneqq&\; t~|~w : \varphi~|~R(w,w')\land w':\varphi~|~R(w,w')\to w':\varphi
  \\[.5ex]
  \mbox{constraint $\chi$}   &\; \coloneqq&\; u \abstractorder t \qquad \abstractorder\in \{\,<,\,\leq,\,>,\,\geq\,\}
\end{array}
\]
If  $w:\varphi$ occurs in   a c-term $u$, we say  that $u$ has label $w$.
A constraint of the form $t \abstractorder t$ is called \emph{atomic};
if $\chi= w:\varphi \abstractorder t$ is non-atomic and $\sharp$ is
the main connective of $\varphi$, we say that $\chi$ is a
\emph{$\sharp$-constraint}.
Constraints where $u$ is   $R(w,w')\land w':\varphi$ or
$R(w,w')\to w':\varphi$ are called
\emph{$R$-constraints}.
Given a multiset of constraints $\G$, by
$\Atm{\G}$ we denote the set of atomic constraints in $\G$.
Let $\M=\stru{W,R,e}$ be a \GWM-model. An $\M$-interpretation of
$\LC$ is a function $\Ical$ mapping labels of $\LC$ to $W$.
We extend $\Ical$ to c-terms as follows:
\[
  \begin{array}{l}
    \Ical(k) = k,\;k\in\{0,1\}
    \qquad
    \Ical( w:\varphi) = e(\Ical(w), \varphi)
    \qquad
    \Ical\left(R(w,w')\right) =  R\left(\Ical(w),\Ical(w')\right)
    \\[1ex]
    \Ical\left(R(w,w') \star w':\varphi\right) =
   R\left(\Ical(w),\Ical(w')\right) \,\star\, e\left(\Ical(w'),\varphi\right),\;\star\in\{\land, \to\}
  \end{array}
\]  
 % \begin{itemize}
 % \item $\Ical(k) = k$, for $k\in\{0,1\}$;
 % \item $\Ical( w:\varphi) = e(\Ical(w), \varphi)$;
 % \item $\Ical\left(R(w,w')\right) =  R\left(\Ical(w),\Ical(w')\right)$;
 % \item $\Ical\left(R(w,w') \star w':\varphi\right) =
 %   R\left(\Ical(w),\Ical(w')\right) \,\star\, e\left(\Ical(w'),
 %     \varphi\right)$, where $\star\in\{\land, \to\}$.
 %  \end{itemize}
Note that, for every c-term $u$, $\Ical(u)$ belongs to $[0,1]$.
 We introduce the relations
$\models_\Ical$ and $\models$, where $\M$ is a \GWM-model, $\Ical$ an
$\M$-interpretation, $\Gamma$ a multiset of constraints.
\begin{itemize}
\item $\M\models_\Ical u \abstractorder t$ iff   $\Ical(u)  \abstractorder \Ical(t)$;
  
\item $\M\models_\Ical \Gamma$ iff $\M\models_\Ical\chi$, for every
  $\chi\in\Gamma$;

\item $\M\models \Gamma$ iff $\M\models_\Ical \Gamma$ for some
  $\M$-interpretation $\Ical$.
\end{itemize}
A \emph{substitution} $\s$ is a function mapping each atomic c-term of
the form $w : p$ or $R(w,w')$ to a rational number in $\Qrange$; $\s$
is extended to all the atomic c-terms by setting $\s(k)=k$, for
$k\in\{0,1\}$, and $\s(w:\bot)=0$.  Let $\Gat$ be a set of atomic
constraints.  By $\s(\Gat)$ we denote the set of constraints obtained
by replacing every atomic c-term $t$ occurring in $\Gat$ with $\s(t)$.
Note that $\s(\Gat)$ is a set of rational constraints of the form $r_1
\abstractorder r_2$, with $r_1$ and $r_2$ in $\Qrange$;
if all the constraint in $\s(\Gat)$ hold, 
$\s$ is a  \emph{solution} to $\Gat$.
The set $\Gat$ is \emph{consistent} iff it admits at least a solution.
% \begin{itemize}[leftmargin=*]
% \item $\Gat$ is \emph{consistent} iff there exists a substitution $\s$
%   such that all the constraints in $\s(\Gat)$ hold; we call $\s$ a
%   \emph{solution} to $\Gat$.
% \end{itemize}
We remark that consistency of $\Gat$ can be checked by a Constraint
Solver over $Q$: one has to abstract  the c-terms $w : p$ and $R(w,w')$ occurring in $\Gat$
by introducing new variables ranging over
$\Qrange$, and then check the consistency of the obtained constraints
by exploiting the solver.



Hereafter, $\lt\in\{<,\leq\}$ and $\gt \in \{>,\geq\}$. The rules of
the calculus $\Calcw$ are displayed in Fig.~\ref{fig:calc1}
(propositional rules) and Fig.~\ref{fig:calc2} (modal rules and
$R$-rules).
%%The axiom rules (namely, the rules without premises) are $\ruleAxat$,
%%$\ruleAxz$ and $\ruleAxo$.  
% It consists of propositional rules, i.e., the axiom rule $\ruleAxat$
% and rules for the propositional connectives,  the modal rules
% $\Box\lt$ and $\Diam\gt$ and rules to handle $R$-constraints.
The \emph{main constraint} of a rule application is the constraint
displayed in the conclusion.  Rules for implication, having a main
constraint of the kind $w:\a\to \b\abstractorder t$, are defined
according to the structure of $\b$.  Let us consider the rule
$\ruleToLTAt$, having main constraint $w:\a\to p < t$ ($p\in\PV\cup\{\bot\}$). The premise
contains the constraints $w:\a > w:p$ and $w:p < t$.  This reflects
the fact that, given a $\GWM$-model $\M$, a world $w$ in $\M$ and an
$\M$-interpretation $\Ical$, if $e(\Ical(w), \a\to p) < \Ical(t)$,
it holds that $e(\Ical(w),\a) > e(\Ical(w),p)$, $e(\Ical(w), \a\to p) =
e(\Ical(w),p)$ and $e(\Ical(w),p) < \Ical(t)$.  This reasoning cannot
be generalized   to any constraint of the form  $w:\a\to \b < t$;
indeed, if $\b\not\in\PV\cup\{\bot\}$, the constraint $w:\a > w:\b$ is not allowed.
This case is covered by rule $\ruleToLT$,
which  introduces a new propositional variable $q$ behaving in
$w$ as $\b$, and $w:\a > w:\b$ is replaced by $w:\a > w: q$.  The
correspondence between $\b$ and $q$ is set by the constraint $w:\b\leq
w:q$, while the converse constraint $w:\b\geq w:q$ can be omitted.
There are two modal rules (see Fig.~\ref{fig:calc2}), the former
having main constraint $w:\Box\a \lt t$ and the latter $w:\Diam\a \gt
t$; both rules introduce a new label $w_1$.  $R$-rules (see
Fig.~\ref{fig:calc2}) are similar to the corresponding propositional
rules.  The definitions of \emph{tree} and \emph{derivation} of
$\Calcw$ are the usual ones (see, e.g.,~\cite{TroSch:00}).  We remark
that $\Calcw$ has the \emph{subformula property}, namely: if $\Tcal$
is a tree of $\Calcw$ having $\G_0$ as root, every formula occurring
in $\Tcal$ inside a constraint (e.g., the formula $\varphi$ in the
constraint $R(w,w')\to w':\varphi$) is a subformula of a formula in
$\G_0$ or a new propositional variable.  By $\provesw{\Gamma}$ we mean
that there exists a derivation of $\Gamma$.  In the rest of this
section we show that $\Calcw$ is a sound and complete refutation
calculus for $\GWL$. Formally, soundness and completeness are defined
as follows, where $\G$ is a multiset of constraints of the form
$w:\varphi\abstractorder t$:

\begin{itemize}
\item (Soundness) if $\provesw{\Gamma}$, then $\M\not\models\G$, for
  every \GWM-model $\M$;
\item (Completeness) if $\M\not\models\G$, for every \GWM-model  $\M$, then $\provesw{\Gamma}$.
\end{itemize}
We stress that, if   rule $\ruleAxat$   checked the
consistency of $\Atm{\G}$ instead of  $\Atmp{\G}$, the calculus $\Calcw$ would  not be complete.
Indeed, let  $\G=\{w : \Box p > w:p,\,w:p\leq 1,\,w:p\geq 1\}$.
Clearly, there is no  \GWM-model   $\M$  such that
$\M\models \G$ hence,  by completeness,  $\G$ must  be  provable in $\Calcw$.
To build a derivation of $\G$,  one can only exploit  rule $\ruleAxat$.
Note that $\Atm{\G}=\{w:p\leq 1,\,w:p\geq 1 \}$
is  consistent (any substitution $\s$ such that $\s(w:p)=1$ is a solution);
thus, if $\ruleAxat$  checked the consistency of  $\Atm{\G}$,
   $\G$ could not be proved.
Instead, $\G$ is proved  since $\ruleAxat$   evaluates
the set $\Atmp{\G} =\Atm{\G}\cup \{1 > w:p\}$, which  is not consistent.
% However, there is no  \GWM-model   $\M=\stru{W,R,e}$  and $\Ical$ such
% that $\M\models_\Ical \G$, since $e(\Ical(w), \Box \a) \leq 1$.
% In $\Calcw$,  $\G$ is provable in $\Calcw$ since rule
% $\ruleAxat$ checks the consistency of $\Atmp{\G} = \{1 > 1\}$.
% Indeed, let $\G=\{w : \Box \a >w:p, w:p \leq 1,
% w:p \geq 1\}$. Clearly, there are no  \GWM-model  $\M=\stru{W,R,e}$  and $\Ical$ such
% that $\M\models_\Ical \G$, otherwise we would get $e(\Ical(w),p)=1$ and
% $e(\Ical(w), \Box \a) >  e(\Ical(w),p)$, namely $e(\Ical(w), \Box \a) > 1$,
% a contradiction.  We point out that
% $\Atm{\G}$ is consistent, but $\G$ is provable in $\Calcw$ since rule
% $\ruleAxat$ checks the consistency of
% $\Atmp{\G} = \{1 > w:p,\,w:p \leq 1,\,w:p \geq 1\}$.






\bigskip 
\inlinetodo{**TO DO???**
An example of derivation is shown in Fig.~\ref{fig:der}.  
In representing derivations, we underline the main constraint of a rule
application; moreover, we omit redundant constraints (e.g.,
constraints of the kind $t \leq t$ and multiple copies of the same
constraint).}


To prove the soundness of $\Calcw$, we need the
following:

%%######### EXAMPLE DERIVATION #########
\begin{figure}[t]
  \centering\small
  ** TO DO **
  % \[
  % \begin{array}{l}
  %   \AXC{}
  %   \RightLabel{$\ruleAxat$}
  %   \UIC{$\G_1$}
  %   % ////////////////////////////////////////////////////////////////////////////////////////
  %   \AXC{}
  %   \RightLabel{$\ruleAxat$}
  %   \UIC{$\G_3$}
  %   % //////////////////
  %   \AXC{}
  %   \RightLabel{$\ruleAxat$}
  %   \UIC{$\G_5$}
  %   % ////////
  %   \AXC{}
  %   \RightLabel{$\ruleAxat$}
  %   \UIC{$\G_6$ }
  %   % ==========================
  %   \RightLabel{$\to \leq$}
  %   \BIC{$\underline{w_1:\neg p \leq c_2},\,w_1:\bot \geq c_2,\, 1 > c_0,\,  w_1:p \leq c_1,\,\D$}
  %   % /////
  %   \AXC{}
  %   \RightLabel{$\ruleAxat$}
  %   \UIC{$\G_4$}
  %   % ===================
  %   \RightLabel{$\to \gt$}
  %   \BIC{$w_1:p \leq c_1,\,\underline{w_1:\neg \neg p > c_0},\,\D$}
  %   % //////////////////
  %   \RightLabel{$\Box\lt$}
  %   \BIC{$\underline{w: \Box p \leq c_1},\,\D$}
  %   % /////////////////////////////////////
  %   \AXC{}
  %   \RightLabel{$\ruleAxat$}
  %   \UIC{$\G_2$}
  %   % ==================================
  %   \insertBetweenHyps{\hskip -5em}
  %   \RightLabel{$\to \gt$}
  %   \BIC{$\underline{w:\neg  \Box p > w:\bot},\,w:\bot \leq c_0,\,w: \Box \neg \neg p > c_0,\,c_0 < 1$ }
  %   % ==========================================================================================
  %   \RightLabel{$\to \leq$}
  %   \insertBetweenHyps{\hskip -7em}
  %   \BIC{$\underline{w:\neg \neg  \Box p \leq c_0},\;w: \Box \neg \neg p > c_0,\;c_0 < 1$}
  %   % -----------------------------------------------------
  %   \RightLabel{$\to <$}
  %   \UIC{$\underline{w: \Box \neg \neg p \to \neg \neg  \Box p < 1}$}
  %   \DP
  %   % #############################################    
  %   \\[14ex]
  %   \D = w:\bot \geq c_1,\,1 > w:\bot,\,w:\bot \leq c_0,\,
  %   w: \Box \neg \neg p > c_0,\,c_0 < 1
  %   \\[.5ex]    
  %   \G_1=c_0 \geq 1,\,   w: \Box \neg \neg p > c_0,\,c_0 < 1
  %   \\[.5ex]    
  %   \G_2 = w:\bot > w:\bot,\,w:\bot \leq c_0,\, w: \Box \neg \neg p > c_0,\,c_0 < 1
  %   \\[.5ex]    
  %   \G_3 =1 \leq c_1,  w:\bot \geq c_1,\,1 > w:\bot,\,w:\bot \leq c_0,\, 1 > c_0,\,c_0 < 1
  %   \\[.5ex]
  %   \G_4 = w_1 : \bot >  c_0,\,w_1:p \leq c_1,\,\D
  %   \\[.5ex]   
  %   \G_5 = c_2\geq 1,\,w_1:\bot \geq c_2,\, 1 > c_0,\,  w_1:p \leq c_1,\,\D
  %   \\[.5ex]    
  %   \G_6 =  w_1 : p > w_1 :\bot,\, w_1 :\bot\leq c_2 ,\,w_1:\bot \geq c_2,\, 1 > c_0,\,  w_1:p \leq c_1,\,\D
  % \end{array}   
  % \]
  \caption{A derivation of $w:...$.}
  \label{fig:der}
\end{figure}

\begin{lemma}\label{lemma:soundRule}
  Let $\rho$ be an instance of a rule of the calculus $\Calcw$, let
  $\Gamma$ be the conclusion of $\rho$ and let $\M$ be a \GWM-model.
  If $\M\models\G$, then there exists a premise $\G'$ of $\rho$ such
  that $\M\models\G'$.
\end{lemma}

\begin{proof}
  Let $\rho$ be the rule $\ruleAxat$; since the rule has no premises,
  we have to show that there is no $\M$ such that $\M\models \G$.  Let
  us assume, by contradiction, that there is a \GWM-model $\M=\stru{W,R,e}$ and an
  $\M$-interpretation $\Ical$ such that $\M\models_\Ical \G$.  If $ w
  : \Box \a > t \in \G$, then $e(\Ical(w), \Box \a) > \Ical(t)$, hence
  $1 > \Ical(t)$, which implies $\M\models_\Ical 1 > t$.  Similarly,
  if $ w : \Diam \a < t \in \G$, then $\M\models_\Ical 0 < t$.  This
  proves that $\M\models_\Ical \Atmp{\G}$.  By exploiting $R$ and  $e$, 
 one can define a solution to $\Atmp{\G}$\footnote{Note that irrational values of $R$ and $e$ 
    must be approximated with rationals.};
  accordingly, $\Atmp{\G}$ is consistent, a contradiction.  We
  conclude that $\M$ does not exist. %
  The cases concerning the other rules are detailed in the
  Appendix~\ref{app:proofs}.
\end{proof}

\begin{proposition}[Soundness]\label{prop:sound}
  \begin{enumerate}[label=(\roman*), ref=(\roman*),leftmargin=*]    
  \item\label{prop:sound:1} If $\provesw{\G}$, then $\M\not\models
    \G$, for every \GWM-model $\M$.
    
  \item\label{prop:sound:2} $\provesw w:\varphi < 1$ implies
    $\varphi\in\GWL$.
  \end{enumerate}
\end{proposition}

\begin{proof} 
  Point~\ref{prop:sound:1} can be  proved by induction on the
  depth of a derivation of $\G$, by exploiting
  Lemma~\ref{lemma:soundRule}.  Let
  us  assume  $\varphi\not\in\GWL$.  Then, there exists a \GWM-model
  $\M=\stru{W,R,e}$ and a world $ w^\star\in W$ such that
  $e(w^\star,\varphi) < 1$.  Let $\Ical$ be an $\M$-interpretation
  mapping $w$ to $w^\star$; it holds that $\M\models_\Ical w:\varphi < 1$. 
  By~\ref{prop:sound:1},
we get $\nprovesw w:\varphi < 1$, and this
proves~\ref{prop:sound:2}. 
\end{proof}





\noindent
We show that $\Calcw$ is \emph{strongly terminating}, namely: there
exists a well-founded relation $\prec$ such that, for every
application $\rho$ of a rule of $\Calcw$, if $\G$ is the conclusion of
$\rho$ and $\G'$ any of the premises of $\rho$, then $\G'\prec \G$.
Equivalently, given a finite multiset $\G$ and repeatedly applying the
rules of $\Calcw$ upwards, proof search eventually halts, no matter
which strategy is used.

To prove strong termination of $\Calcw$, we introduce the relation $\precc$ between multisets of constraints. 
The weight $\wg{u}$ of a constraint $u$, the weight $\wg{\G}$ of a multiset
of constraints $\G$, and the size $\size{\varphi}$ of a formula
$\varphi$ are defined as follows:
\begin{center}
  \begin{tabular}{p{40ex}p{40ex}}\small
    \[
    \begin{array}{l}
      \wg{u} \;=\;
      \begin{cases}
        0 & \mbox{if $u$ is an atomic c-term}
        \\
        \size{\varphi}& \mbox{if $u = w:\varphi$}
        \\
        \size{\varphi} + 1&\begin{minipage}[t]{20ex}
          if $u = R(w,w')\star w':\varphi$\par~~~with $\star\in \{\land,\to\}$
        \end{minipage}
      \end{cases}
      \\[7ex]
      \wg{u\abstractorder t}\,=\,\wg{u}
      \hspace{3em}
      \wg{\G}\,=\,\sum_{\chi\in\Gamma} \wg{\chi}
    \end{array}    
    \]
    &\small
    \[
    \size{\varphi} \;=\;
    \begin{cases}
      0 & \mbox{if $\varphi\in\PV\cup\{\bot\}$}
      \\
      \size{\a} + \size{\b} + 1 &\begin{minipage}[t]{20ex}
        if $\varphi=\a\star\b$\par~~~with $\star\in \{\land,\lor,\to\}$
    \end{minipage}
    \\[2ex]
    \size{\a} + 2 & \mbox{if $\varphi=\Box\a$ or $\varphi=\Diam\a$}
    \end{cases}
    \]
  \end{tabular}
\end{center}
% We extend $\wgname$ as follows ($\G$ is a  multiset of constraints):
% \[\small
% \begin{array}{l}
%   \wg{u} \;=\;
%   \begin{cases}
%     0 & \mbox{if $u$ is an atomic c-term}
%     \\
%     \size{\varphi} &     \mbox{if $u = w:\varphi$}
%     \\
%     \size{\varphi} + 1  &\mbox{if $u = R(w,w')\land w':\varphi$ or $u = R(w,w')\to w':\varphi$ }
%   \end{cases}
%   \\[6ex]
%   \wg{u \abstractorder t}  \,=\, \wg{u}
%   \hspace{5em}
%   \wg{\G}\,=\,\sum_{\chi\in\Gamma} \wg{\chi}
% \end{array}    
% \]

\noindent
Let $\G[w]$ (multiset of constraints) and $\sizem{\G}$ (multiset of natural numbers) be defined as follows:
\[
\G[w] \,=\,\{\,
u  \abstractorder t\in\G~|~\mbox{$\lab{u} = w$}
\,\} 
\qquad
\sizem{\G} \,=\, \{\, \wg{\G[w]}~|~\mbox{$w$ occurs in $\G$}    \,\}
\]
Given the finite multisets of natural numbers
$\Theta_1$ and $\Theta_2$,  we set:% \\[.5ex]
\[
\Theta_1 \precm \Theta_2
% \qquad\IFF\qquad
\quad\mbox{iff}\quad
\Theta_1\neq
\Theta_2 \;\land\; \left(\; \forall k_1 \in \Theta_1\setminus
  \Theta_2.\ \exists k_2\in \Theta_2 \setminus \Theta_1.\ k_1 < k_2
  \;\right).
\]  
The relation $\precm$ is a multiset order, hence $\precm$ is
well-founded (see~\cite{BaaderN:98}, Th.~2.5.5 and Lemma 2.5.6).
Given the multisets of constraints $\G_1$ and $\G_2$,
we set $\G_1 \precc \G_2$ iff  $\sizem{\G_1} \precm\sizem{\G_2}$. 
% We introduce the following well-founded relation $\precc$ between finite multisets
% of constraints:
% \[
% \G_1 \precc \G_2 \quad\mbox{iff}\quad \sizem{\G_1} \precm\sizem{\G_2}. 
% \]  
% We can prove (see App.~\ref{app:proofs}):

% \begin{lemma}\label{lemma:rulesDec}
%   Let $\rho$ be an application of a rule of the calculus $\Calcw$, let
%   $\Gamma$ be the conclusion of $\rho$ and  $\G'$ any of the premises of $\rho$.
%   Then,  $\G'\precc \G$.
% \end{lemma}




\begin{proposition}\label{prop:term}
    The calculus $\Calcw$ is strongly terminating.
\end{proposition}
\begin{proof}
  The relation $\precc$ is well-founded and can be used to instantiate
  $\prec$ in the definition of strong termination (see
  Lemma~\ref{lemma:rulesDec} in the Appendix).
\end{proof}

Accordingly, any backward proof search strategy for $\Calcw$
terminates.  Following~\cite{FerFioFio:2013,FioFer:2021jlc}, we focus
on strategies where failure is certified by countermodels, i.e.: if
proof search for a multiset of constraints $\G$ fails, a \GWM-model
$\M$ such that $\M\models \G$ can be built; we call $\M$ a
\emph{countermodel} for $\G$, since $\M$ ascertains that $\G$ is not
provable in $\Calcw$ (by its soundness).  Let $\G$ be a multiset of
constraints:

\begin{itemize}
\item $\G$ is \emph{reduced} iff no rule of $\Calcw$ can be backward
  applied to $\G$; thus, every non-atomic constraint in $\G$ has the
  form $w:\Box \a \gt t$ or $w:\Diam \a \lt t$.

\item $\G$ is \emph{plain} iff only a modal rule can be backward
  applied to $\G$; thus, every non-atomic constraint in $\G$ has the
  form $w:\Box \a \abstractorder t$ or $w:\Diam \a \abstractorder t$.
\end{itemize}
An application of a modal rule is plain iff its conclusion is plain.
By $\app{\chi}{\G_k}{\G_{k+1}}$ we mean that $\G_{k+1}$ is a premise
of an application of a rule $\rho$ of $\Calcw$ having conclusion
$\G_k$ and main constraint $\chi$, with $\chi\in\G_k$.  A branch
$\Bcal$ is a sequence $\stru{\G_0,\dots,\G_n}$ such that, for every $0
\leq k < n$, there is $\chi_k$ such that
$\app{\chi_k}{\G_k}{\G_{k+1}}$; thus, $\Bcal$ represents a branch of a
tree of $\Calcw$.  The branch $\Bcal$ is \emph{saturated} iff the
following holds:
  
\begin{itemize}[leftmargin=*]
\item $\G_0$ is a finite set of constraints of the form
  $w:\varphi\abstractorder t$;

\item  $\G_n$ is reduced;

\item every application of a modal rule in $\Bcal$ is plain.
  % for every $k\geq 0$, if $\G_k$ is the conclusion of a modal rule,
  % then $\G_k$ is p-reduced.
\end{itemize}

\noindent
We show how to construct a countermodel for $\G_0$ from $\G_n$.  Let
$\Gat$ be a finite consistent set of atomic constraints and let $\s$
be a solution to $\Gat$.  By $\Mod{\Gat,\sigma}$ we denote the
$\GM$-model $\M=\stru{W,R,e}$ such that $W$ is the set of labels
occurring in $\Gat$ and:
\[\small
\begin{array}{l}
  R(w,w') \;=\;
  \begin{cases}
    \s(R(w,w')) & \mbox{if $R(w,w')$ occurs in $\Gat$}  
    \\
    0 & \mbox{otherwise}
  \end{cases}
  \qquad
  e(w,p) \;=\;
  \begin{cases}
    \s(w:p) & \mbox{if $w:p$ occurs in $\Gat$}  
    \\
    0 & \mbox{otherwise}
  \end{cases}
\end{array}
\]
We remark that $\M$ is a discrete $\GWM$-model.  A
$\Mod{\Gat,\sigma}$-interpretation $\Ical$ is \emph{canonical} iff
$\Ical(w)=w$ for every label $w$ occurring in $\Gat$.

In Lemma~\ref{lemma:count} we show that if
$\Bcal=\stru{\G_0,\dots,\G_n}$ is saturated and $\s$ is a solution to
$\Atmp{\G_n}$, then $\Mod{\Atmp{\G_n},\s}$ is a countermodel for
$\G_0$.

\begin{figure}[t]
  \centering
    \[\small
  \begin{array}{c}
    % Gamma_7
    \AXC{$w_1:p<R(w_0,w_1),\;\; w_1:p\leq c, \;\;  w_1 : q \geq R(w_0,w_1),\;\; 1> c,\;\;  R(w_0,w_1)\leq c,\;\;\D$}     
    % ----------------------------------------------
    % Gamma_6
    \RightLabel{$R\land\lt(l)$}
    \UIC{$w_1:p<R(w_0,w_1),\;\; w_1:p\leq c, \;\;  w_1 : q \geq R(w_0,w_1),\;\; 1> c,\;\; \underline{R(w_0,w_1)\land w_1 : q \leq c},\;\;\D$}
    % ----------------------------------------------
    % Gamma_5
    \RightLabel{$\lor\gt(r)$}
    \UIC{$w_1:p<R(w_0,w_1),\;\; w_1:p\leq c, \;\;  \underline{w_1 :p\lor q \geq R(w_0,w_1)},\;\; 1> c,\;\; R(w_0,w_1)\land w_1 : q \leq c,\;\;\D$}
    % -------------------------------------------------------------
    % Gamma_4
    \RightLabel{$R\to\gt(l)$}
    \UIC{$w_1:p<R(w_0,w_1),\;\; w_1:p\leq c, \;\;  \underline{R(w_0,w_1)\to w_1 :p\lor q > c},\;\; R(w_0,w_1)\land w_1 : q \leq c,\;\;\D$} 
    % ----------------------------------------------
    % Gamma_3
    \RightLabel{$R\to\leq(r)$}
    \UIC{$\underline{R(w_0,w_1)\to w_1 :p \leq c},\;\;  R(w_0,w_1)\to w_1 :p\lor q > c,\;\; R(w_0,w_1)\land w_1 : q \leq c,\;\;\D$} 
    % ----------------------------------------------
    % Gamma_2
    \RightLabel{$\Box\lt$}
    \UIC{$w_0:  \Box(p\lor q) > c,\;\; \underline{w_0 : \Box p \leq  c},\;\; w_0: \Diam q  \leq  c,\;\;  c   < 1$} 
    % ----------------------------------------------
    % Gamma_1
    \RightLabel{$\lor\lt$}
    \UIC{$w_0:  \Box(p\lor q) > c,\;\;\underline{w_0 : (\Box p\lor \Diam q) \leq c},\;\; c  < 1$}
    % ----------------------------------------------
    % Gamma_0
    \RightLabel{$\to <$}
    \UIC{\underline{$w_0:  \Box(p\lor q) \to (\Box p\lor \Diam q)  < 1$}}
    \DP
    \\[18ex]
    % ###############################################
    c = w_0:q_0\qquad\D\;=\;\{\,     w_0:  \Box(p\lor q) > c,\;\;  w_0: \Diam q  \leq  c,\;\;  c   < 1 \,\}    
  \end{array}
  \]  
  \vspace{-2ex}
  \caption{A saturated branch.}
  \label{fig:saturatedbranch}
\end{figure}


\begin{example}\label{ex:satBrancj}
  Let $\varphi=\Box(p\lor q) \to (\Box p\lor \Diam q)$ be an instance
  of $(Cr)$.  The tree shown in Fig.\ref{fig:saturatedbranch}
  corresponds to a saturated branch $\Bcal=\stru{\G_0,\dots,\G_7}$,
  where $\G_0=\{w_0:\varphi < 1\}$ is the root and $\G_7$ is the top
  multiset.  When the applied rule has two premises, the annotation
  $(l)$ (left) or $(r)$ (right) specifies the selected one; in every
  rule application, the main constraint is underlined.

  The set $\Atmp{\G_7}$ consists of the following atomic constraints:
  \[\small
  w_1:p<R(w_0,w_1),\;\; w_1:p\leq c, \;\; w_1 : q \geq R(w_0,w_1),\;
  1> c,\;\; R(w_0,w_1)\leq c,\;\; c < 1
  \]
  The set $\Atmp{\G_7}$ is consistent; a solution to $\Atmp{\G_7}$ is
  any substitution $\s$ over $\Atmp{\G_7}$ such that
  \[
  \sigma(w_1:p)\, < \,\sigma\left(R(w_0,w_1)\right)\,\leq\, \sigma(c) \,< 1
  \qquad
  \sigma\left(R(w_0,w_1)\right)\,\leq \, \sigma(w_1: q) %\,\leq\, 1
  \]
  For every solution $\s$, the \GWM-model $\M=\Mod{\Atmp{\G_7},\s}$
  satisfies $\M \models_\Ical w_0 :\varphi < 1$, where $\Ical$ is a
  canonical $\M$-interpretation; as a consequence, $\M$ is a countermodel for
  $\varphi$, witnessing that $\varphi\not\in\GWL$.
  Note that the worlds of $\M$ are $w_0$ and $w_1$ and,
  as expected, $\M$ is not crisp, since $0 < \sigma(R(w_0,w_1)) < 1$.
    A solution
  $\s^\star$ is obtained by setting $\sigma^\star(w_1:p) = 0.5$ and
  $\sigma^\star(R(w_0,w_1))= \sigma^\star(c) = \sigma^\star(w_1:q) =
  0.6$.  The model $\Mod{\Atmp{\G_7},\s^\star}$ is essentially the
  same as the one displayed in Fig.~\ref{fig:countGW}, the only
  difference is in the evaluation of $q_0$ in $w_0$.
  \EndEx
\end{example}



\begin{lemma}\label{lemma:count}
  Let $\Bcal=\stru{\G_0,\dots,\G_n}$ be a saturated branch and $\s$ a
  solution to $\Atmp{\G_n}$.  Let $\M=\Mod{\Atmp{\G_n},\s}$, $\Ical$ a
  canonical $\M$-interpretation  and $\chi\in
  \bigcup_{k\in\{0,\dots,n\}}\G_k$.  Then, $\M\models_\Ical \chi$.
\end{lemma}


\begin{proof}
  Since $\M=\Mod{\Atmp{\G_n},\s}$, with $\s$ is a solution to $\Atmp{\G_n}$,
  it holds that:
  \begin{enumerate}[label=(\arabic*), ref=(\arabic*),leftmargin=*]
    % (Atmp)
  \item\label{lemma:count:atmp}
    $\M\models_\Ical \Atmp{\G_n}$.
  \end{enumerate}
  We prove  $\M\models_\Ical \chi$ by induction on $\wg{\chi}$;
  we assume $\M=\stru{W,R,e}$.
  Let $\wg{\chi}=0$; then $\chi$ is atomic, hence $\chi\in\G_n$,
  and $\M\models_\Ical \chi$ by~\ref{lemma:count:atmp}.
  
  Let $\wg{\chi}>0$ and let us assume that  $\chi\not\in \G_n$.
  Then, there exists $k \geq  0$ such that
  $\app{\chi}{\G_k}{\G_{k+1}}$. 
  We proceed by a case analysis on the structure of $\chi$.
  
  
  %%%%%%%%%%%%%%%% AND GT
  Let $\chi= w:\a\land \b\gt t$.  Since $\app{\chi}{\G_k}{\G_{k+1}}$,
  the applied rule is $\land\gt$, hence $w:\a\gt t\in \G_{k+1}$ and
  $w:\b \gt t\in \G_{k+1}$.  By the induction hypothesis
  $\M\models_\Ical w:\a\gt t$ and $\M\models_\Ical w:\b \gt t $, and
  this implies $\M\models_\Ical w:\a\land \b\gt t$.  The case $\chi=
  w:\a\lor \b\lt t$ is similar.

  %%%%%%%%%%%%%%%% AND LT
  Let $\chi= w:\a\land \b\lt t$.  Since $\app{\chi}{\G_k}{\G_{k+1}}$,
  the applied rule is $\land\lt$, hence $w:\a\lt t \in \G_{k+1}$ or
  $w:\b\lt t \in \G_{k+1}$.  By the induction hypothesis
  $\M\models_\Ical w:\a\lt t $ or $\M\models_\Ical w:\b\lt t $, hence
  $\M\models_\Ical w:\a\land \b\lt t$.  The case $\chi= w:\a\lor \b\gt
  t$ is similar.
 
 
  %%%%%%%%%%%%%%%% IMP LTS  
  Let $\chi= w:\a\to \b < t$; we only consider the case
  $\b\not\in\PV\cup\{\bot\}$.  Since $\app{\chi}{\G_k}{\G_{k+1}}$, the
  applied rule is $\to<$, hence:
 \begin{enumerate}[label=(\Alph*), ref=(\Alph*),leftmargin=*]
   % (A)
 \item\label{lemma:count:impless:1}
   $\{\,w:\a > w : q,\, w:\b \leq w : q,\,w: q < t\,\} \subseteq \G_{k+1}$. 
 \end{enumerate}
 By the induction hypothesis,
 for every constraint $\chi'$  displayed in~\ref{lemma:count:impless:1},
 $\M\models_\Ical \chi'$, hence:
 \[
 e(w,\a) > e(w,q) \qquad e(w,\b) \leq e(w,q) \qquad e(w,q) < \Ical(t)
 \]  
 Since $e(w,\a) > e(w,\b)$, we get  $e(w,\a\to \b) = e(w,\b)$, hence  $e(w,\a\to \b) < \Ical(t)$,
 which implies $\M\models_\Ical  w:\a\to \b < t  $.
 
 
 %%%%%%%%%%%%%%%% IMP LTE
 Let $\chi= w:\a\to \b \leq t$; we only consider the case
 $\b\not\in\PV\cup\{\bot\}$.  Since $\app{\chi}{\G_k}{\G_{k+1}}$, the
 applied rule is $\to\leq$, hence one of the followings two subcases
 holds:
 
 \begin{enumerate}[label=(B\arabic*), ref=(B\arabic*),leftmargin=*]
   % (B1)
 \item\label{lemma:count:impleq:1}
   $t\geq 1\in \G_{k+1}$;
   
   % (B2)
 \item\label{lemma:count:impleq:2}
   $\{\,w:\a> w :q,\,w:\b\leq w:q,\,w:q\leq t\,\} \subseteq \G_{k+1}$.
 \end{enumerate}
 Let us assume that~\ref{lemma:count:impleq:1} holds.  By the
 induction hypothesis, we get $\M\models_\Ical t\geq 1$, and this
 implies $\M\models_\Ical w:\a\to \b \leq t$.  Let us assume
 that~\ref{lemma:count:impleq:2} holds.  By the induction hypothesis,
 for every constraint $\chi'$ in~\ref{lemma:count:impleq:2},
 $\M\models_\Ical \chi'$, hence:
 \[
 e(w,\a) > e(w,q) \qquad e(w,\b) \leq e(w,q) \qquad e(w,q) \leq \Ical(t)
 \]  
 Since $e(w,\a) > e(w,\b)$, we get $e(w,\a\to \b) = e(w,\b)$, hence
 $e(w,\a\to \b) \leq \Ical(t)$, which implies $\M\models_\Ical w:\a\to
 \b \leq t$.


 %%%%%%%%%%%%%%%% IMP GT
 Let $\chi= w:\a\to \b \gt t$; we only consider the case
 $\b\not\in\PV\cup\{\bot\}$.  Since $\app{\chi}{\G_k}{\G_{k+1}}$, the
 applied rule is $\to\gt$, hence one of the followings two subcases
 holds:
 
 \begin{enumerate}[label=(C\arabic*), ref=(C\arabic*),leftmargin=*]
   % (C1)
 \item\label{lemma:count:implgt:1}
   $\{\,w:\a \leq w:q,\,w:\b\geq  w:q,\,1\gt t\,\}\subseteq\G_{k+1}$;
   
   % (C2)    
 \item\label{lemma:count:implgt:2}
   $w:\b\gt t\in\G_{k+1}$.
 \end{enumerate}
 
 \noindent
 Assume~\ref{lemma:count:implgt:1}.  By the induction hypothesis, for
 every constraint $\chi'$ in~\ref{lemma:count:implgt:1},
 $\M\models_\Ical \chi'$, hence:
 \[
 e(w,\a) \leq  e(w,q) \qquad e(w,\b) \geq e(w,q) \qquad 1 \gt \Ical(t)
 \]  
 Since $e(w,\a)\leq e(w,\b)$, we get $e(w,\a\to \b)=1$, hence
 $e(w,\a\to \b)\gt \Ical(t)$, which implies $\M\models_\Ical\chi$.  In
 case~\ref{lemma:count:implgt:2}, by the induction hypothesis we get
 $\M\models_\Ical w:\b\gt t$, namely $e(w,\b)\gt\Ical(t)$.  Since
 $e(w,\a\to\b)\geq e(w,\b)$, we get $e(w,\a\to\b)\gt\Ical(t)$, hence
 $\M\models_\Ical w:\a\to \b \gt t$.  The cases concerning
 $R$-constrains can be proved as the corresponding propositional
 cases.


 %%%%%%%%%%%%%%%% BOX LT
 Let $\chi= w:\Box\a\lt t$.  Since $\app{\chi}{\G_k}{\G_{k+1}}$, the
 applied rule is $\Box\lt$, hence:
 \begin{enumerate}[label=(\Alph*), ref=(\Alph*),leftmargin=*,start=4]
   
   % (D)
 \item\label{lemma:count:boxlt}
   % $\chi_1\in \G_{k+1}$, where $\chi_1\,=\,R(w,w_1)\to w_1:\a\lt t$.
   $R(w,w_1)\to w_1:\a\lt t \, \in \G_{k+1}$.
 \end{enumerate}
 Let $r_1$ be the value of $R(w,w_1)\to e(w_1:\a)$.  By the induction
 hypothesis $\M\models_\Ical R(w,w_1)\to w_1:\a\lt t $, hence $r_1\lt\
 \Ical(t)$.  Since $e(w,\Box\a)\leq r_1$, we get $e(w,\Box\a) \lt
 \Ical(t)$, and this implies $\M\models_\Ical w:\Box\a\lt t$.  The
 case $\chi= w:\Diam\a\gt t$ is similar.

 %%%%%%%%%%%%%%%%%%%%%%% BOX GT 
 It remains to consider the cases where $\wg{\chi}>0$ and $\chi\in
 \G_n$, thus $\chi= w:\Box\a\gt t$ or $\chi= w:\Diam\a\lt t$.  Let
 $\chi= w:\Box\a\gt t$; we show that:

 \begin{enumerate}[label=(\Alph*), ref=(\Alph*),leftmargin=*,start=5]
   % (E)
 \item\label{lemma:count:boxgt}
   $\M\models_\Ical R(w,w') \to w':\a\gt t$, for every  $w'\in W$.
 \end{enumerate}
 Let $w'\in W$; we show that $\M\models_\Ical R(w,w') \to w':\a\gt t$.
 Let us assume that $R(w,w')$ does not occur in $\G_n$.  By definition
 of $\M$, the value of $R(w,w')$ is 0, hence we have to show that
 $\M\models_\Ical 1 \gt t$.  The case where $\gt$ is $\geq$ is
 immediate.  Let $\gt$ be $>$.  Since $\chi\in\G_n$, we get $1
 >t\in\Atmp{\G_n}$, thus $\M\models_\Ical1 > t$
 by~\ref{lemma:count:atmp}.  Let us assume that $R(w,w')$ occurs in
 $\G_n$.  Then, there exists $k > 0$ be such that $R(w,w')$ is
 introduced in $\G_{k}$ by an application of a modal rule $\rho$,
 namely:
 \begin{itemize}
 \item $\app{\chi'}{\G_{k-1}}{\G_k}$, with $\chi' = w :\Box \varphi \lt t'$ or  $\chi' = w :\Diam\varphi \gt t'$,
   and the new label introduced by $\rho$ is  $w'$.
 \end{itemize}
 Since $\Bcal$ is saturated, $\G_{k-1}$ is plain, hence the non-atomic
 constraints in $\G_{k-1}$ are modal.  As a consequence, no rule
 applied in the sub-branch $\stru{\G_{k-1},\dots,\G_n}$ can introduce
 in the premise new constraints $u\abstractorder t'$ such that $u$ has
 label $w$; this implies that $\chi\in \G_{k-1}$.  By definition of
 $\rho$, we get $\Phibd{\G_{k-1},w,w'}\subseteq \G_{k}$, hence
 $R(w,w') \to w':\a\gt t \in\G_{k}$. By the induction hypothesis, we
 get $\M\models_\Ical R(w,w') \to w':\a\gt t$, and this concludes the
 proof of~\ref{lemma:count:boxgt}.  Let $e(w,\Box\a)=r$.  By the witnessing condition,
  there exists $w^\star\in W$ such that the value of
 $R(w,w^\star) \to e(w^\star,\a)$ is $r$.  By~\ref{lemma:count:boxgt},
 $r\gt \Ical(t)$, namely $e(w,\Box\a)\gt \Ical(t)$; we conclude
 $\M\models_\Ical w:\Box\a\gt t$.  The case $\chi= w:\Diam\a\lt t$ is
 similar.
\end{proof}


\noindent
As a consequence, we get:
\begin{proposition}\label{prop:satBranch}
  Let $\Bcal=\stru{\G_0,\dots,\G_n}$ be a saturated branch.
  \begin{enumerate}[label=(\roman*), ref=(\roman*),leftmargin=*]
  \item\label{prop:satBranch:1} For every solution  $\s$  to
    $\Atmp{\G_n}$, $\Mod{\Atmp{\G_n},\s}$ is a discrete
    countermodel for $\G_0$.
      
  \item\label{prop:satBranch:2} If $\G_0=\{ w : \varphi < 1\}$, then
    $\varphi\not\in\GWL$.
  \end{enumerate}
\end{proposition}


Let $\Bs$ be a backward proof search strategy for $\Calcw$; we say
that $\Bs$ is \emph{plain} iff all the modal rule applications
performed by $\Bs$ are plain.


\begin{lemma}\label{lemma:search}
  Let $\Bs$ be a plain proof search strategy for $\Calcw$ and let
  $\G_0$ be a finite multiset of constraints of the form
  $w:\varphi\abstractorder t$.  If $\Bs$ fails to prove $\G_0$, then a
  discrete countermodel for $\G_0$ can be built.
\end{lemma}

\begin{proof}
  Assume that $\Bs$ fails.  By tracing the computation, we can build
  an open branch $\Bcal=\stru{\G_0,\dots,\G_n}$.  Since $\Bs$ is
  plain, the branch $\Bcal$ is saturated.
Let $\s$ be  a solution to $\Atmp{\G_n}$;
by Prop.~\ref{prop:satBranch}, 
  $\Mod{\Atmp{\G_n},\s}$ is a discrete countermodel for $\G_0$.
\end{proof}

\noindent
%%We remark that $\Bs$ does not need to implement backtracking.
As a corollary, we get the completeness of $\Calcw$:



\begin{proposition}[Completeness]\label{prop:compl}
  Let $\G$ be a finite multiset of constraints of the form
  $w:\varphi\abstractorder t$.  If $\nprovesw{\G}$ then there exists a 
  discrete countermodel for $\G$.
  %%\begin{enumerate}[label=(\roman*), ref=(\roman*),leftmargin=*]
  %%\item\label{prop:compl:1} If  $\nprovesw{w: \varphi < 1}$, then there exists a
  %%  discrete countermodel for $\varphi$.
  %%\item\label{prop:compl:2}
  %%  If $\nprovesw{w : \varphi < 1}$, then  $\varphi\not\in\GWL$.
  %%\item\label{prop:compl:3} If $\varphi\not\in\GWL$, then then there
  %%  exists a discrete countermodel for $\varphi$.
  %%\end{enumerate}
\end{proposition}
% \begin{proof}
%   Point~\ref{prop:compl:1} follows from Lemma~\ref{lemma:search}.
%   Point~\ref{prop:compl:2} follows from~\ref{prop:compl:1},
%   since a countermodel for $w :\varphi < 1$ certifies that $\varphi\not\in\GWL$.
%   Point~\ref{prop:compl:3} follows from~\ref{prop:compl:1} since, by soundness of $\Calcw$,
%   $\varphi\not\in\GWL$ implies $\nprovesw{w : \varphi < 1}$.
% \end{proof}


% \noindent
% Accordingly, if $\provesw{w : \varphi < 1}$ then $\varphi\in\GWL$.

By Prop.~\ref{prop:compl}, $\GWL$ has the finite model property. Let
$\varphi\not\in\GWL$ and let $\M$ be the discrete countermodel
extracted from an open branch having root $w_0:\varphi<1$.  One can
easily prove that the depth of $\M$ is bounded by $\size{\varphi}$ and
that every world of $\M$ has at most $\size{\varphi}$ R-successors;
this implies that the size of $\M$ is
$O(\size{\varphi}^{\size{\varphi}})$ (see
Th.~4.1~\cite{BilkovaFK:22}).  Note that it is not possible to build
$\M$ one branch at a time, since the constraints generated during the
expansion of a branch have global validity and must be kept throughout
the construction.  However, by adapting the procedure described in
Th.~4.2~\cite{BilkovaFK:22}, one can prove that the decision problem
for $\GWL$ is in PSPACE.

We have implemented both the proof search procedure and the
countermodel extraction in the JTabWb
framework~\cite{FerrariFF:17a}. The prover, named
\texttt{gwref}~\cite{gwrefProver}, performs a standard backward
depth-first proof search, leveraging the JTabWb engine complemented
with the Java implementation of the $\Calcw$ rules and of a plain
proof-search strategy. The consistency of atomic constraints is
checked using the Choco-solver Java library~\cite{ChocoSolver:2025}.

To our knowledge, the only other prover available for modal fuzzy
logic is mNiBLoS~\cite{Vidal:16}, an SMT-based solver designed for
continuous t-norm-based logics, including G\"odel-Dummett
Logic. However, mNiBLoS does not support the logic $\GWL$.


We remark that, if in the above proof search procedure we impose to
the solver to find solutions $\s$ such that $\s(R(w,w'))\in\{0,1\}$
for every pair of labels $w,w'$ we get a calculus for $\GWCL$, the
crisp extension of $\GWL$ \cite{FerFioRod:2025}. On the contrary, is
not trivial to adapt our calculus to the non witnessed G\"odel modal
logics.



%%% Local Variables: 
%%% mode: latex
%%% TeX-master: "goedelModalLogicWitnessNonCrisp"
%%% End: 
