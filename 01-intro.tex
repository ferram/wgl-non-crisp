\section{Introduction ** OLD **}\label{sec:intro}

% \inlinetodo{ {\bf Stress in the introduction}.  Additionally, we are
%   confident that the relationship between Gödel-Kripke models and
%   linear bi-relational models is of considerable interest to the CADE
%   community, as it provides a means to compare a calculus defined
%   under one semantics with that defined under the other.
  
%   Finally, regarding our second contribution, we would like to mention
%   that while the bi-relational semantics may appear orthogonal to the
%   primary goals of the conference, we believe that this perspective
%   enriches the logical landscape by placing $\GWCL$ within the broader
%   framework of intuitionistic modal logics. Our construction of a
%   mutual correspondence (Lemma 5 and Prop. 6) offers a novel and
%   accessible characterization, bypassing technical machinery such as
%   universal algebra, and shows that $\GWCL$ can be embedded naturally
%   within a known semantic tradition.  

%   Kripke semantics with fuzzy accessibility. The standard approach considers relations
%   valued in [0,1], reflecting degrees of accessibility. This leads to rich logics which
%   can capture uncertainty and vagueness in agent systems.
% }

One of the most important challenges in Knowledge Representation and
Reasoning is to set up a homogeneous framework to represent and
combine the notions of incompleteness, uncertainty and vagueness. In
this sense, different many-valued modal logics have become a lingua
franca for, among others, the specification and analysis of knowledge
and communication in multi-agent and distributed systems. Notably,
G\"odel Modal Logic $\GKL$ has provided an adequate logical foundation
for these tasks, so it has been widely studied in the last decade.

In general, $\GKL$ semantics rests on G{\"o}del Kripke models
(\emph{\GM-models}), the generalization of the classical Kripke
semantics for modal logics, where both propositions at each world and
the accessibility relation are valued in the standard G{\"o}del
algebra $[0,1]$, namely the algebraic semantics of the well-known
intermediate G\"odel-Dummett Logic. The minimal logics over
{\GM-models} have been deeply investigated by Caicedo and Rodriguez in
\cite{CaiRod2010,CaiRod2015}.  A peculiar subclass of this kind of
models is obtained by restricting to crisp models
(\emph{\GCM-models}), where the accessibility relation only takes
classical values, i.e., it is valued in $\{0,1\}$.
%%A peculiar subclass of this kind of
%%models is obtained by restricting to crisp models
%%(\emph{\GCM-models}), where the accessibility relation only takes
%%classical values, i.e., it is valued in $\{0,1\}$.
The logic $\GCL$ arising from \GCM-models has been characterized in
\cite{RodriguezV:21}. More general approaches, focusing mainly on
finite residuated lattices, have been developed by
Fitting~\cite{Fi92a,Fi92b}, Priest~\cite{Pr08b}, and Bou et
al.~\cite{BoEsGoRo11}, and for other relevant fuzzy logics
in~\cite{HaTe13,ViEsGo16}.

In this line of research, we introduce a logic called $\GWCL$, which
is semantically characterized by \emph{witnessed} G\"odel crisp Kripke
models (\GWCM-models); in \GWCM-models, the value of a modal formula
$\Box \a$ or $\Diamond \a$ in a world $w$ coincides with (thus, is
witnessed by) the value of $\a$ in a world accessible from $w$. A
crucial point is that $\GWCL$ has the finite model property (if
$\varphi\not\in\GWCL$, than there exists a finite \GWCM-model falsifying
$\varphi$), while such a property fails for the bi-modal logics
arising from {\GM-models} and from {\GCM-models}.

Another important aspect to consider is that the semantical approach
to modal many-valued logics considered here differs from the
alternative significant framework for modal substructural logics
studied, for instance, in~\cite{Kam02,O93,Re93b}.  This latter
framework is based on a syntactic definition of the logics, obtained
by extending substructural logics with modalities governed by some of
the usual axioms and rules for the modalities of classical modal
logics. The resulting logics are complete with respect to a
bi-relational Kripke semantics, where the evaluation function typical
of \GM-models is replaced by additional relations, similar to the
semantics used in intuitionistic modal logics.  A prominent
contribution of this paper is to establish a simple connection between
the semantics of $\GWCL$ and a bi-relational semantics, showing that
$\GWCL$ can be naturally embedded within the well-established semantic
tradition of intuitionistic modal logics. This connection provides a
means to compare a calculus defined under one semantics with that
defined under the other.

As for practical reasoning in AI, research has primarily focused on
developing efficient calculi for the aforementioned many-valued
logics.  Metcalfe and Olivetti in \cite{MetOli1,MetOli2} have provided
variants of a hypersequent structure called \emph{sequent of relations
  calculus} (firstly introduced by Baaz and Ferm\"uller in
\cite{BaazF99}) for several fragments of $\GKL$. Galmiche and Salhi
use a similar approach in \cite{GalSal2010} for studying a family of
G\"odel hybrid logics.  Burns in~\cite{BURNS2020} surveys the
approaches based on hypersequents; in particular, it is maintained
that relational hypersequents provide a unified proof theory for many
modal logics with cut-free complete treatments.  Here we propose for
$\GWCL$ a different approach inspired by the tableau calculus
presented in~\cite{BilkovaFK:22}, but overcoming some flaws in it.

To sum up, the main result of this paper is the definition of a new
calculus $\Calcc$ which is sound and complete for the logic $\GWCL$.
We design a terminating decision procedure for $\Calcc$ with
countermodel generation, consisting in a standard backward proof
search strategy for which we provide an implementation, called
\texttt{gwcref} (accessible at~\cite{gwcrefProver}).
%%, in the Java framework JTabWb~\cite{FerrariFF:17a}. 
Finally we explore a novel and simple connection between the
$\GWCM$-models and the bi-relational Kripke semantics. For the omitted
proofs, we refer the reader to the appendix available online at the
URL referenced in~\cite{gwcrefProver}.

%%% Local Variables: 
%%% mode: latex
%%% TeX-master: "goedelModalLogicWitnessNonCrisp"
%%% End: 
