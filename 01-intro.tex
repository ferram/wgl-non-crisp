\section{Introduction}\label{sec:intro}


G\"odel modal logics, $\GKL$s, provide a natural framework for
reasoning about graded notions of incompleteness, uncertainty, and
vagueness by means of necessity and possibility operators evaluated
over truth degrees. When interpreted over the real unit interval,
G\"odel logic captures a well-behaved notion of implication based on
residuation, which makes it particularly suitable for applications in
knowledge representation, as well as in multi-agent and distributed
systems.


In general, the semantics of $\GKL$ is based on G\"odel Kripke models
(\emph{\GM-models}), a natural generalization of classical Kripke
semantics in which both the valuation of propositions at each world
and the accessibility relation take values in the standard G\"odel
algebra $[0,1]$, i.e.\ the algebraic semantics of the intermediate
G\"odel--Dummett logic. The minimal logics over {\GM-models} were
thoroughly investigated by Caicedo and Rodr\'iguez in
\cite{CaiRod2010,CaiRod2015}. In this setting, the interpretation of
the modal operators is defined by means of infima and suprema computed
over (possibly infinite) sets of accessible worlds. However, these
bounds need not be attained: the infimum may be strictly smaller than
every truth value in the set, and dually for the
supremum. Consequently, the truth value of a modal formula may depend
on limit constructions rather than on concrete worlds of the model.

This phenomenon has important logical consequences. For instance,
\cite{CaiRod2010} shows that the formula $\Box \neg \neg \varphi \to
\neg \neg \Box \varphi$ is valid in all finite G\"odel Kripke models
but fails in a suitable infinite model. Hence, the underlying logic
does not enjoy the finite model property, and semantic arguments may
require genuinely infinitary behavior.

Witnessed models provide a principled way to overcome these
difficulties. By requiring that the relevant infima and suprema are
realized by actual accessible worlds, witnessed semantics ensures that
existential and universal behavior is supported by explicit
witnesses. In this way, the evaluation of modal operators more closely
mirrors the behavior of guarded quantification in first-order logic
and restores a constructive reading of the semantics. Each modal
statement is justified by concrete worlds rather than by purely
asymptotic approximations.

The notion of witnessed models has appeared in several related
contexts. For example, in~\cite{HC2006} an axiomatization is given for
witnessed semantics in fuzzy predicate logics. Since quantified
statements can often be translated into modal ones, this connection is
conceptually robust. Along these lines, H\'ajek considers
in~\cite{HajekS5} the axioms $\Diamond(\varphi \to \Box \varphi)$,
$\Diamond(\Diamond \varphi \to \varphi)$ and proves that the resulting
logic $S5(G)^w$ is strongly complete with respect to witnessed
universal Kripke models.

Witnessed semantics has also proved fruitful in applied settings. In
the context of Fuzzy Description Logics, \cite{BDP2014} studies a
system based on witnessed G\"odel semantics and shows that it enjoys
the finite model property and a decidable satisfiability problem with
exponential-time complexity. More recently, \cite{RV26} provides an
axiomatization for modal G\"odel logic interpreted over witnessed
Kripke models.

In general, witnessed models tend to guarantee compactness-like
properties and frequently yield finite model properties and
decidability results. These features are crucial for automated
reasoning, verification, and implementation, since they allow one to
replace infinitary semantic arguments with finite combinatorial
ones. For these reasons, studying G\"odel modal logics over witnessed
models not only clarifies their semantic foundations but also
strengthens their connections with well-behaved first-order fragments
that play a central role in computer science, thereby enhancing both
their theoretical robustness and their applicability.

From a proof-theoretic perspective, witnessed semantics aligns more
closely with syntactic calculi. By eliminating pathological behaviors
caused by unattained infima and suprema, it avoids limit-based
counterexamples and yields a more constructive semantics. This tight
correspondence between semantic witnesses and syntactic derivations
naturally suggests the development of a complete proof system. In
particular, the existence of witnesses enables canonical model
constructions that are finite or finitely generated, supports
filtration techniques, and allows completeness proofs to proceed by
standard algebraic or canonical methods. Therefore, establishing a
sound and complete calculus for G\"odel modal logics over witnessed
models is not only feasible but also conceptually justified, providing
a proof-theoretic counterpart to their improved semantic behavior and
paving the way for effective automated reasoning procedures.



% \inlinetodo{ {\bf Stress in the introduction}.  Additionally, we are
%   confident that the relationship between Gödel-Kripke models and
%   linear bi-relational models is of considerable interest to the CADE
%   community, as it provides a means to compare a calculus defined
%   under one semantics with that defined under the other.
  
%   Finally, regarding our second contribution, we would like to mention
%   that while the bi-relational semantics may appear orthogonal to the
%   primary goals of the conference, we believe that this perspective
%   enriches the logical landscape by placing $\GWCL$ within the broader
%   framework of intuitionistic modal logics. Our construction of a
%   mutual correspondence (Lemma 5 and Prop. 6) offers a novel and
%   accessible characterization, bypassing technical machinery such as
%   universal algebra, and shows that $\GWCL$ can be embedded naturally
%   within a known semantic tradition.  

%   Kripke semantics with fuzzy accessibility. The standard approach considers relations
%   valued in [0,1], reflecting degrees of accessibility. This leads to rich logics which
%   can capture uncertainty and vagueness in agent systems.
% }

% One of the most important challenges in Knowledge Representation and
% Reasoning is to set up a homogeneous framework to represent and
% combine the notions of incompleteness, uncertainty and vagueness. In
% this sense, different many-valued modal logics have become a lingua
% franca for, among others, the specification and analysis of knowledge
% and communication in multi-agent and distributed systems. Notably,
% G\"odel Modal Logic $\GKL$ has provided an adequate logical foundation
% for these tasks, so it has been widely studied in the last decade.

% In general, $\GKL$ semantics rests on G{\"o}del Kripke models
% (\emph{\GM-models}), the generalization of the classical Kripke
% semantics for modal logics, where both propositions at each world and
% the accessibility relation are valued in the standard G{\"o}del
% algebra $[0,1]$, namely the algebraic semantics of the well-known
% intermediate G\"odel-Dummett Logic. The minimal logics over
% {\GM-models} have been deeply investigated by Caicedo and Rodriguez in
% \cite{CaiRod2010,CaiRod2015}.  A peculiar subclass of this kind of
% models is obtained by restricting to crisp models
% (\emph{\GCM-models}), where the accessibility relation only takes
% classical values, i.e., it is valued in $\{0,1\}$.
% %%A peculiar subclass of this kind of
% %%models is obtained by restricting to crisp models
% %%(\emph{\GCM-models}), where the accessibility relation only takes
% %%classical values, i.e., it is valued in $\{0,1\}$.
% The logic $\GCL$ arising from \GCM-models has been characterized in
% \cite{RodriguezV:21}. More general approaches, focusing mainly on
% finite residuated lattices, have been developed by
% Fitting~\cite{Fi92a,Fi92b}, Priest~\cite{Pr08b}, and Bou et
% al.~\cite{BoEsGoRo11}, and for other relevant fuzzy logics
% in~\cite{HaTe13,ViEsGo16}.

% In this line of research, we introduce a logic called $\GWCL$, which
% is semantically characterized by \emph{witnessed} G\"odel crisp Kripke
% models (\GWCM-models); in \GWCM-models, the value of a modal formula
% $\Box \a$ or $\Diamond \a$ in a world $w$ coincides with (thus, is
% witnessed by) the value of $\a$ in a world accessible from $w$. A
% crucial point is that $\GWCL$ has the finite model property (if
% $\varphi\not\in\GWCL$, than there exists a finite \GWCM-model falsifying
% $\varphi$), while such a property fails for the bi-modal logics
% arising from {\GM-models} and from {\GCM-models}.

% Another important aspect to consider is that the semantical approach
% to modal many-valued logics considered here differs from the
% alternative significant framework for modal substructural logics
% studied, for instance, in~\cite{Kam02,O93,Re93b}.  This latter
% framework is based on a syntactic definition of the logics, obtained
% by extending substructural logics with modalities governed by some of
% the usual axioms and rules for the modalities of classical modal
% logics. The resulting logics are complete with respect to a
% bi-relational Kripke semantics, where the evaluation function typical
% of \GM-models is replaced by additional relations, similar to the
% semantics used in intuitionistic modal logics.  A prominent
% contribution of this paper is to establish a simple connection between
% the semantics of $\GWCL$ and a bi-relational semantics, showing that
% $\GWCL$ can be naturally embedded within the well-established semantic
% tradition of intuitionistic modal logics. This connection provides a
% means to compare a calculus defined under one semantics with that
% defined under the other.

% As for practical reasoning in AI, research has primarily focused on
% developing efficient calculi for the aforementioned many-valued
% logics.  Metcalfe and Olivetti in \cite{MetOli1,MetOli2} have provided
% variants of a hypersequent structure called \emph{sequent of relations
%   calculus} (firstly introduced by Baaz and Ferm\"uller in
% \cite{BaazF99}) for several fragments of $\GKL$. Galmiche and Salhi
% use a similar approach in \cite{GalSal2010} for studying a family of
% G\"odel hybrid logics.  Burns in~\cite{BURNS2020} surveys the
% approaches based on hypersequents; in particular, it is maintained
% that relational hypersequents provide a unified proof theory for many
% modal logics with cut-free complete treatments.  Here we propose for
% $\GWCL$ a different approach inspired by the tableau calculus
% presented in~\cite{BilkovaFK:22}, but overcoming some flaws in it.

% To sum up, the main result of this paper is the definition of a new
% calculus $\Calcc$ which is sound and complete for the logic $\GWCL$.
% We design a terminating decision procedure for $\Calcc$ with
% countermodel generation, consisting in a standard backward proof
% search strategy for which we provide an implementation, called
% \texttt{gwcref} (accessible at~\cite{gwcrefProver}).
% %%, in the Java framework JTabWb~\cite{FerrariFF:17a}. 
% Finally we explore a novel and simple connection between the
% $\GWCM$-models and the bi-relational Kripke semantics. For the omitted
% proofs, we refer the reader to the appendix available online at the
% URL referenced in~\cite{gwcrefProver}.

%%% Local Variables: 
%%% mode: latex
%%% TeX-master: "goedelModalLogicWitnessNonCrisp"
%%% End: 
