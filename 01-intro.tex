
\section{Introduction}\label{sec:intro}

G\"odel modal logics, $\GKL$s, provide a natural framework for reasoning
about graded notions of incompleteness, uncertainty, and vagueness by
means of necessity and possibility operators evaluated over truth
degrees. When interpreted over the real unit interval, G\"odel logic
captures a well-behaved notion of implication based on residuation,
which makes it particularly suitable for applications in knowledge
representation, as well as in multi-agent and distributed systems.


In general, the semantics of $\GKL$ is based on G\"odel Kripke models
(\emph{\GM-models}), a natural generalization of classical Kripke
semantics in which both the valuation of propositions at each world
and the accessibility relation take values in the standard G\"odel
algebra $[0,1]$, i.e.\ the algebraic semantics of the intermediate
G\"odel--Dummett logic. The minimal logics over {\GM-models} were
thoroughly investigated by Caicedo and Rodr\'iguez in
\cite{CaiRod2010,CaiRod2015}.  In this setting, the interpretation of
the modal operators is defined by means of infima and suprema computed
over (possibly infinite) sets of accessible worlds. However, these
bounds need not be attained: the infimum may be strictly smaller than
every truth value in the set, and dually for the
supremum. Consequently, the truth value of a modal formula may depend
on limit constructions rather than on concrete worlds of the model.


This phenomenon has significant logical consequences. For instance,
\cite{CaiRod2010} shows that the formula $\Box \neg \neg \varphi \to
\neg \neg \Box \varphi$ is valid in all finite G\"odel Kripke models,
but fails in a suitable infinite model. Hence, the underlying logic
does not enjoy the finite model property, and semantic arguments may
require genuinely infinitary behavior.

Witnessed models provide a principled way to overcome these
difficulties. By requiring that the relevant infima and suprema be
realized by actual accessible worlds, witnessed semantics ensures that
existential and universal behavior is supported by explicit
witnesses. In this way, the evaluation of modal operators more closely
mirrors the behavior of guarded quantification in first-order logic
and restores a constructive reading of the semantics. Each modal
statement is thus justified by concrete worlds rather than by purely
asymptotic approximations.

The notion of witnessed models has appeared in several related
contexts. For example, in~\cite{HC2006} an axiomatization is given for
witnessed semantics in fuzzy predicate logics. Since quantified
statements can often be translated into modal ones, this connection is
conceptually well grounded. Along these lines, H\'ajek considers
in~\cite{HajekS5} the axioms $\Diamond(\varphi \to \Box \varphi)$ and
$\Diamond(\Diamond \varphi \to \varphi)$, and proves that the resulting
logic $S5(G)^w$ is strongly complete with respect to witnessed
universal Kripke models.

Witnessed semantics has also proved fruitful in applied settings. In
the context of Fuzzy Description Logics, \cite{BDP2014} studies a
system based on witnessed G\"odel semantics and shows that it enjoys
the finite model property and a decidable satisfiability problem with
exponential-time complexity. More recently, \cite{RV26} provides an
axiomatization for modal G\"odel logic interpreted over witnessed
Kripke models.

In general, witnessed models tend to ensure compactness-like
properties and frequently yield the finite model properties and
decidability results. These features are crucial for automated
reasoning, verification, and implementation, since they allow one to
replace infinitary semantic arguments with finite combinatorial
ones. For these reasons, studying G\"odel modal logics over witnessed
models not only clarifies their semantic foundations but also
strengthens their connections with well-behaved fragments of
first-order logic that play a central role in computer science, thereby
enhancing both their theoretical robustness and their applicability.

From a proof-theoretic perspective, witnessed semantics aligns more
closely with syntactic calculi. By eliminating pathological behaviors
arising from unattained infima and suprema, it avoids limit-based
counterexamples and yields a more constructive semantics. This tight
correspondence between semantic witnesses and syntactic derivations
naturally suggests the development of a complete proof system. In
particular, the existence of witnesses enables canonical model
constructions that are finite or finitely generated, supports
filtration techniques, and allows completeness proofs to proceed by
standard algebraic or canonical methods. Therefore, establishing a
sound and complete calculus for G\"odel modal logics over witnessed
models is not only feasible but also conceptually justified, providing
a proof-theoretic counterpart to their improved semantic behavior and
paving the way for effective automated reasoning procedures.


According to the above considerations, in this paper we introduce the
logic $\GWL$, which is semantically characterized by \emph{witnessed}
G\"odel Kripke models, a restriction of $\GM$-models in which the
value of a modal formula $\Box \a$ or $\Diamond \a$ at a world $w$
coincides with (and is thus witnessed by) the value of $\a$ at some
world accessible from $w$.  We then introduce a refutation calculus
$\Calcw$ for $\GWL$ and establish its soundness and completeness with
respect to witnessed models. On the basis of $\Calcw$, we design a
terminating decision procedure with countermodel generation, based on
a standard backward proof-search strategy. As a direct consequence of
this decision procedure, $\GWL$ enjoys the finite model property,
since every non-provable formula admits a finite countermodel
effectively extracted from a failed derivation.  Finally, we provide
an implementation of the procedure, called \texttt{gwref} (available
at~\cite{gwrefProver}). We reamrk that our calculus is inspired to the
one presented in \cite{FerFioRod:2025} for $\GWCL$, the crisp variant
of $\GWL$, which, in turn, is inspired by the tableau calculus
presented in~\cite{BilkovaFK:22} for a logics extending $\GWCL$.

%%, in the Java framework JTabWb~\cite{FerrariFF:17a}. 
%%For the omitted proofs, we refer the reader to the appendix available
%%online at the URL referenced in~\cite{gwrefProver}.





%%% Local Variables: 
%%% mode: latex
%%% TeX-master: "goedelModalLogicWitnessNonCrisp"
%%% End: 
