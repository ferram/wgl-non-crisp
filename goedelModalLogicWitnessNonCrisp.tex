%\documentclass[envcountsame,runningheads]{llncs} 
%\documentclass[submission,copyright,creativecommons]{eptcs}
%\providecommand{\event}{AiML 2026} 
%%
%%
%%%%%%%%%%% >>>>>>>>>>>>>>>>>>> AiML style config
\documentclass[submission,copyright,creativecommons]{eptcs}
\providecommand{\event}{AiML 2026} 

% AiML macros (for writing theorems)
\usepackage{aiml26}

\usepackage{iftex}

\ifpdf
  \usepackage{underscore}         % Only needed if you use pdflatex.
  \usepackage[T1]{fontenc}        % Recommended with pdflatex
\else
  \usepackage{breakurl}           % Not needed if you use pdflatex only.
\fi

%%%%%%%%%%% <<<<<<<<<<<<<<<< AiML style config

\usepackage[T1]{fontenc}
\usepackage[misc,geometry]{ifsym}
\usepackage{stmaryrd}
\usepackage{amsmath}
\usepackage{amssymb}
\usepackage{xspace}
\usepackage{mathtools}
\usepackage{bussproofs}\EnableBpAbbreviations     
\usepackage{colortbl}
\usepackage{tikz}
\usetikzlibrary{shapes.geometric, arrows,arrows.meta}
\usepackage{float}
\usepackage{pifont}




\newcommand{\cmark}{\ding{51}}%
\newcommand{\xmark}{\ding{55}}%

%\usepackage{pgfplots}
%\pgfplotsset{compat=1.15} 

\usepackage{xifthen}%
\usepackage{url}
\usepackage[inline]{enumitem}
\setlist{topsep=.3em,left=0pt}
\usepackage{xcolor}
\usepackage{array}
\usepackage{hyperref}
\usepackage{wrapfig}
\usepackage{mathdots} 


% %% todonotes
\usepackage[color=green!20,linecolor=red]{todonotes}
\newcommand{\inlinetodo}[1]{\todo[inline,size=\small]{#1}}
\newcommand{\tm}[1]{{\color{magenta} #1}} 


%% LOGICS
\newcommand{\GL}{\mathbf{G}}
\newcommand{\GKL}{\mathbf{GK}}
\newcommand{\GWCL}{\mathbf{GW}^\mathrm{c}}
\newcommand{\GWL}{\mathbf{GW}}
\newcommand{\GCL}{\mathbf{G}^\mathrm{c}}
\newcommand{\GWCM}{\mbox{$\mathrm{GW}^\mathrm{c}$}}
\newcommand{\GCM}{\mbox{$\mathrm{G}^\mathrm{c}$}}
\newcommand{\GM}{\mbox{$\mathrm{G}$}}
\newcommand{\GWM}{\mbox{$\mathrm{GW}$}}

\newcommand{\Calcc}{\Ccal_{\mathrm{GW}^\mathrm{c}}}
\newcommand{\Calcw}{\Ccal_{\mathrm{GW}}}
\newcommand{\KL}{\mathbf{K}}
\newcommand{\IKL}{\mathbf{K}_\Box}
\newcommand{\IPL}{\mathbf{IPL}\xspace}
\newcommand{\CPL}{\mathbf{CPL}\xspace}

\newcommand{\LBM}{\mathbf{Lbm}}%%{\Logof{\K_{\GWCM}}}



\newcommand{\KGL}{\mathbf{KG}^2_\mathrm{fb}}
\newcommand{\TKGL}{\Tcal(\KGL)}


\newcommand{\ruleToLTAt}{\to_{\mathrm{At}}<}
\newcommand{\ruleToLT}{\to<}
\newcommand{\ruleToLEQAt}{\to_{\mathrm{At}}\leq}
\newcommand{\ruleToLEQ}{\to\leq}
\newcommand{\ruleToGAt}{\to_{\mathrm{At}}\gt}
\newcommand{\ruleToG}{\to\gt}

%% CALCULI
% \newcommand{\GT}{\mathbf{G3i}\xspace}
% \newcommand{\GTI}{\mathsf{G3i}\xspace}
% \newcommand{\GT}{\mathsf{G3iSL}^+_{\Box}\xspace}
% \newcommand{\GTT}{\mathsf{G3}\xspace}
% \newcommand{\GTGiessena}{\mathsf{G3iSL}^{a}_{\Box}\xspace}
% \newcommand{\GTGiessen}{\mathsf{G3iSL}_{\Box}\xspace}
% \newcommand{\GFGiessen}{\mathsf{G4iSL}_{\Box}\xspace}
% \newcommand{\GFI}{\mathsf{G4i}\xspace}
% \newcommand{\GF}{\mathsf{G4}\xspace}
% \newcommand{\GFGore}{\mathsf{G4iSLt}}
% \newcommand{\Gbu}{\mathsf{GbuSL}_{\Box}\xspace}
% \newcommand{\RJ}{\mathsf{RbuSL}_{\Box}\xspace}

%% Connectives
\newcommand{\IFF}{\Longleftrightarrow}
\newcommand{\tto}{\leftrightarrow}
\newcommand{\Diam}{\lozenge}

\newcommand{\Phibd}[1]{\Phi^{\Box,\Diam}(#1)}
\newcommand{\Phizo}[1]{\Phi^{0,1}(#1)}

\newcommand{\LC}{\Lcal_c}


\newcommand{\Qrange}{[0,1]_{\mathrm{Q}}}

\newcommand{\precm}{\prec_{\mathrm{m}}} 
\newcommand{\precc}{\prec_{\mathrm{c}}}



\newcommand{\PV}{\Vcal}
\newcommand{\Sf}[1]{\mathrm{Sf}(#1)}
\newcommand{\Sfm}[1]{\mathrm{Sf}^-(#1)}
\newcommand{\Gat}{\G_\mathrm{at}}
\newcommand{\Gstar}{\G^\ast}
\newcommand{\lab}[1]{\mathrm{label}(#1)}

\newcommand{\reass}[3]{#1[#2 \coloneqq #3]}
\newcommand{\xmax}{x_{\mathrm{max}}}
\newcommand{\xmin}{x_{\mathrm{min}}}
\newcommand{\Logof}[1]{\mathrm{L}(#1)}


% new arrow tip
\newdimen\arrowsize 
\pgfarrowsdeclare{arcs'}{arcs'}{...} 
{ 
  \arrowsize=0.5pt 
  \advance\arrowsize by .5\pgflinewidth 
  \pgfsetdash{}{0pt} % do not dash 
  \pgfsetroundjoin   % fix join 
  \pgfsetroundcap    % fix cap 
  \pgfpathmoveto{\pgfpoint{-4\arrowsize}{4\arrowsize}} 
  \pgfpatharc{180}{270}{4\arrowsize} 
  \pgfpatharc{90}{180}{4\arrowsize} 
  \pgfusepathqstroke 
}



%
\makeatletter
\newbox\xrat@below
\newbox\xrat@above
\newcommand{\coimparrow}[2][]{%
  \setbox\xrat@below=\hbox{\ensuremath{\scriptstyle #1}}%
  \setbox\xrat@above=\hbox{\ensuremath{\scriptstyle #2}}%
  \pgfmathsetlengthmacro{\xrat@len}{max(\wd\xrat@below,\wd\xrat@above)+.65em}%
  \mathrel{\tikz [>-,>=arcs',baseline=-.5ex,line width=.6pt]
                 \draw (0,0) -- node[below=-2pt] {\box\xrat@below}
                                node[above=-2pt] {\box\xrat@above}
                       (\xrat@len,0) ;}}
\makeatother
\newcommand{\coimpl}{\mathrel{\;\coimparrow[\phantom{\text{}}]{}}}
\newcommand{\invcoimpl}{\mathrel{\reflectbox{$\;\coimparrow[\phantom{\text{}}]{}$}}}





%\newcommand{\gt}{\succsim}
%\newcommand{\lt}{\precsim}
\newcommand{\gt}{\rhd}
\newcommand{\lt}{\lhd}


%% Math
\newcommand{\stru}[1]{\langle #1 \rangle} % structure
\newcommand{\abstractorder}{\,\blacktriangledown\,}%{\bigtriangledown}


%% Relations
\newcommand{\proves}[3]{{#2}\,\vdash_{#1}\, #3}
\newcommand{\nproves}[3]{{#2}\,\nvdash_{#1}\, #3}
\newcommand{\provesc}[1]{\vdash_{\mathrm{GW}^\mathrm{c}} #1}
\newcommand{\nprovesc}[1]{\nvdash_{\mathrm{GW}^\mathrm{c}} #1}
\newcommand{\provesw}[1]{\vdash_{\mathrm{GW}} #1}
\newcommand{\nprovesw}[1]{\nvdash_{\mathrm{GW}} #1}
\newcommand{\forcing}{\Vdash}
\newcommand{\nforcing}{\nVdash}


% %% FUNCTIONS
\newcommand{\size}[1]{|#1|}
\newcommand{\sizem}[1]{||#1||}
\newcommand{\depth}[1]{\mathrm{d}(#1)}
\newcommand{\wgname}{\mathrm{wg}}
\newcommand{\wg}[1]{\mathrm{wg}(#1)}
\newcommand{\Lab}[1]{\mathrm{Lab}(#1)}
\newcommand{\Rel}[1]{\mathrm{Rel}(#1)}
\newcommand{\Atm}[1]{\mathrm{At}(#1)}
\newcommand{\Atmp}[1]{\mathrm{At}^{+}\!(#1)}
\newcommand{\Int}[1]{\mathrm{I}(#1)}
\newcommand{\length}[1]{\mathrm{length}(#1)}
\newcommand{\Rn}{\mathrm{Rn}}
\newcommand{\Rnof}[1]{\Rn(#1)}
\newcommand{\Bs}{\mbox{\textsc{Bs}}}
\newcommand{\Bsof}[1]{\Bs(#1)}
 

 
 \newcommand{\Modname}{\mathrm{Mod}}
 \newcommand{\Mod}[1]{\Modname(#1)}
\newcommand{\app}[3]{#2\mapsto_{#1} #3}


%% Kripke models
\newcommand{\K}{\mathcal{K}}
\newcommand{\Val}{V}{%\vartheta}
\newcommand{\Valof}[1]{\Val(#1)}
\newcommand{\M}{\mathfrak{M}}

%% Regular sequents
% \newcommand{\seq}[2]{
%   \ifthenelse{\isempty{#1}}{}{#1}\, \Rightarrow
%   \ifthenelse{\isempty{#2}}{}{#2}
% }


%% Rule names 
\newcommand{\ruleAxat}{\mathrm{Ax}\xspace}
%% \newcommand{\ruleAxz}{\mathrm{Ax}_0\xspace}
%% \newcommand{\ruleAxo}{\mathrm{Ax}_1\xspace}


%% GREEKS
\renewcommand{\a}{\alpha}
\renewcommand{\b}{\beta}
\newcommand{\g}{\gamma}
\renewcommand{\d}{\delta}
\renewcommand{\phi}{\varphi}
\newcommand{\s}{\sigma}
\newcommand{\D}{\Delta}
\newcommand{\G}{\Gamma}


%% CALLIGRAPHICS
\newcommand{\Acal}{\mathcal{A}}
\newcommand{\Bcal}{\mathcal{B}}
\newcommand{\Ccal}{\mathcal{C}}
\newcommand{\Dcal}{\mathcal{D}}
\newcommand{\Ecal}{\mathcal{E}}
\newcommand{\Fcal}{\mathcal{F}}
\newcommand{\Gcal}{\mathcal{G}}
\newcommand{\Hcal}{\mathcal{H}}
\newcommand{\Kcal}{\mathcal{K}}
\newcommand{\Ical}{\mathcal{I}}
\newcommand{\Jcal}{\mathcal{J}}
\newcommand{\Lcal}{\mathcal{L}}
%\newcommand{\Mcal}{\mathcal{M}}
\newcommand{\Ncal}{\mathcal{N}}
\newcommand{\Pcal}{\mathcal{P}}
\newcommand{\Qcal}{\mathcal{Q}}
\newcommand{\Rcal}{\mathcal{R}}
\newcommand{\Scal}{\mathcal{S}}
\newcommand{\Tcal}{\mathcal{T}}
\newcommand{\Vcal}{\mathcal{V}}
\newcommand{\Wcal}{\mathcal{W}}
\newcommand{\Xcal}{\mathcal{X}}
\newcommand{\Ycal}{\mathcal{Y}}
\newcommand{\Zcal}{\mathcal{Z}}


%% VARI
\newcommand{\ov}[1]{\overline{#1}}
\newcommand{\refp}[1]{(\ref{#1})}
\newcommand{\EndEx}{\mbox{}~\hfill$\Diamond$}
\newcommand{\squareforqed}{\hbox{$\blacksquare$}} % redefine box for QED
\newcommand{\diamondforendexmp}{\hbox{$\blacklozenge$}} % define diamond for EndExmp
\newcommand{\smallfont}{\fontsize{7pt}{9pt}\selectfont}
\newcommand{\medfont}{\fontsize{9pt}{10pt}\selectfont}

\newcommand{\MODIFICA}[1]{\colorbox{yellow}{#1}\normalcolor}

\RequirePackage{tcolorbox}
\definecolor{tcbx_Yellow_Bright}{RGB}{250, 250, 142}
\definecolor{tcbx_Yellow_Dark}{RGB}{230, 230, 0}%208, 242, 39}
%\newenvironment{MODIFICA}{\begin{minipage}}{}
\newtcolorbox{tcyellowbox}[2][]{
  beforeafter skip=1\baselineskip,
  % colbacktitle=red!10!white,
  colbacktitle=tcbx_Yellow_Dark,
  %toptitle=0.1cm,
  %bottomtitle=0.2cm,
  top=0.2cm,
  bottom=0.2cm,
  left=0cm,
  %boxsep=5mm,
  %leftupper=18cm,
  colframe=tcbx_Yellow_Dark, 
  colback=tcbx_Yellow_Bright,
  halign lower=flush right,
  fonttitle=\bfseries,
  fontupper=\upshape,
  fontlower=\upshape,
  coltitle=black,
  %center title,
  title={#2},
  #1
}



% % Manual theorems for Appendix (parameter is inner text i.e. number, description)
% \newtheorem{theoremrefinner}{Theorem}
% \newenvironment{theoremref}[1]{%
%   \renewcommand\thetheoremrefinner{#1.}%
%   \theoremrefinner
% }{\endtheoremrefinner}

\newtheorem{lemmarefinner}{Lemma}
\newenvironment{lemmaref}[1]{%
  \renewcommand\thelemmarefinner{#1.}%
  \lemmarefinner
}{\endlemmarefinner}


\newtheorem{proprefinner}{Proposition}
 \newenvironment{propref}[1]{%
   \renewcommand\theproprefinner{#1.}%
   \proprefinner
}{\endproprefinner}



%%%%%%%%%%%%%%%%%%%%%%%%%%%%%%%%%%%%%%%%%%%%%%%%%%%%%%%%%%%%%%%%%%%
%% algorithm2e
% \usepackage[linesnumbered,commentsnumbered,figure]{algorithm2e}
% \let\oldnl\nl% Store \nl in \oldnl
% \newcommand{\nonl}{\renewcommand{\nl}{\let\nl\oldnl}}% Remove line number for one line
% \newcommand{\ass}{\;\leftarrow\;}
% \SetKwProg{Proc}{procedure}{}{end}  % define a procedure 
% \SetKwProg{uProc}{procedure}{}{}  % define a procedure 
% \SetKwFunction{Searchbu}{SearchBU}
% \SetKwData{Refs}{Refs}
% \SetKwInOut{Precond}{precondition}
% \SetKwInOut{Output}{output}
% \SetKwInOut{Input}{input}
% \SetAlFnt{\small}
% \SetAlCapFnt{\small}
% \SetAlCapNameFnt{\small}
% \SetAlCapHSkip{0pt}
% \IncMargin{-\parindent}
% \SetKwInOut{Input}{input}
% \SetKwInOut{Output}{output}
% \SetKwInOut{Assumption}{assumption}
% \SetKwBlock{Loop}{loop}{end}
% \SetKwFor{uFor}{for}{do}{} % For without end


%%%%%%%%%%%%%%%%%%%%%%%%%%%%%%%%%%%%%%%%%%%%%%%%%%%%%%%%%%%%%%%%%%


\title{A G\"odel Modal Logic Over Witnessed Models}
%\authorrunning{M. Ferrari et al.}
\author{
  Mauro Ferrari$^{*}$%%\orcidID{0000-0002-7904-1125}
  \institute{Dep.~of Theoretical and Applied Sciences\\
    Universit\`a degli Studi dell'Insubria\\
    Varese, Italy.}
  \email{mauro.ferrari@uninsubria.it}
  \and
  Camillo Fiorentini%%\orcidID{0000-0003-2152-7488}
  \institute{
    Dep.~of Computer Science\\
    Universit\`a degli Studi di Milano\\
    Milano, Italy}
  \email{fiorentini@di.unimi.it}
  \and
  Paolo Giardini
  \institute{Dep.~of Theoretical and Applied Sciences\\
    Universit\`a degli Studi dell'Insubria\\
    Varese, Italy.}
  \email{pcgiardini@uninsubria.it}
  \and
  Ricardo Oscar Rodriguez%%\inst{3}\orcidID{0000-0001-7551-2877}}
  \institute{
    UBA-FCEyN, Dep.~De Computación\\
    Buenos Aires, Argentina}
  \email{ricardo@dc.uba.ar}
}


\newcommand{\titlerunning}{A G\"odel Modal Logic Over Witnessed  Models}
\newcommand{\authorrunning}{M. Ferrari, C. Fiorentini, P. Giardini \& R.O. Rodriguez}


\begin{document}
\maketitle


%% AIML \url{https://events.illc.uva.nl/aiml2026/}, 29 June – 3 July
%% 2026, Amsterdam Abstract long papers submission: 20 February 2026
%% Full papers submission: 27 February 2026 
%%
%% At most 12 pages
%% (references excluded), with an optional technical appendix of up to
%% 5 pages, together with a plain-text abstract of 100-200 words
\begin{abstract}
  We introduce $\GWL$, a G\"odel modal logic based on Kripke models in
  which the value of a modal formula is witnessed by an accessible
  world. This witnessed semantics eliminates the limit-based phenomena
  that preclude the finite model property in the usual Kripke
  semantics for G\"odel modal logics, thereby yielding a more
  constructive semantic framework. We present a sound and complete
  refutation calculus for $\GWL$ and design a terminating
  backward proof-search procedure with countermodel generation. As a
  direct consequence of this procedure, $\GWL$ enjoys the finite model
  property.
\end{abstract}



\section{Introduction}\label{sec:intro}

G\"odel modal logics, $\GKL$s, provide a natural framework for reasoning
about graded notions of incompleteness, uncertainty, and vagueness by
means of necessity and possibility operators evaluated over truth
degrees. When interpreted over the real unit interval, G\"odel logic
captures a well-behaved notion of implication based on residuation,
which makes it particularly suitable for applications in knowledge
representation, as well as in multi-agent and distributed systems.


In general, the semantics of $\GKL$ is based on G\"odel Kripke models
(\emph{\GM-models}), a natural generalization of classical Kripke
semantics in which both the valuation of propositions at each world
and the accessibility relation take values in the standard G\"odel
algebra $[0,1]$, i.e.\ the algebraic semantics of the intermediate
G\"odel--Dummett logic. The minimal logics over {\GM-models} were
thoroughly investigated by Caicedo and Rodr\'iguez
in~\cite{CaiRod2010,CaiRod2015}.  In this setting, the interpretation
of the modal operators is defined by means of infima and suprema
computed over (possibly infinite) sets of accessible worlds. However,
these bounds need not be attained: the infimum may be strictly smaller
than every truth value in the set, and dually for the
supremum. Consequently, the truth value of a modal formula may depend
on limit constructions rather than on concrete worlds of the model.
%%
This phenomenon has significant logical consequences. For instance,
\cite{CaiRod2010} shows that the formula $\Box \neg \neg \varphi \to
\neg \neg \Box \varphi$ is valid in all finite G\"odel Kripke models,
but fails in a suitable infinite model (see
Fig.\ref{fig:nonWitn}). Hence, the underlying logic does not enjoy the
finite model property, and semantic arguments may require genuinely
infinitary behavior.

Witnessed models provide a principled way to overcome these
difficulties. By requiring that the relevant infima and suprema be
realized by actual accessible worlds, witnessed semantics ensures that
existential and universal behavior is supported by explicit
witnesses. In this way, the evaluation of modal operators more closely
mirrors the behavior of guarded quantification in first-order logic
and restores a constructive reading of the semantics. Each modal
statement is thus justified by concrete worlds rather than by purely
asymptotic approximations.

The notion of witnessed models has appeared in several related
contexts. For example, in~\cite{HC2006} an axiomatization is given for
witnessed semantics in fuzzy predicate logics. Since quantified
statements can often be translated into modal ones, this connection is
conceptually well grounded. Along these lines, H\'ajek considers
in~\cite{HajekS5} the axioms $\Diamond(\varphi \to \Box \varphi)$ and
$\Diamond(\Diamond \varphi \to \varphi)$, and proves that the resulting
logic $S5(G)^w$ is strongly complete with respect to witnessed
universal Kripke models.

Witnessed semantics has also proved fruitful in applied settings. In
the context of Fuzzy Description Logics, \cite{BDP2014} studies a
system based on witnessed G\"odel semantics and shows that it enjoys
the finite model property and a decidable satisfiability problem with
exponential-time complexity. More recently, \cite{RV26} provides an
axiomatization for modal G\"odel logic interpreted over witnessed
Kripke models.

In general, witnessed models tend to ensure compactness-like properties and
often yield the finite model property and decidability
results. These features are crucial for automated reasoning,
%%verification, and implementation, 
since they allow infinitary semantic arguments to be replaced by
finite combinatorial ones. For these reasons, studying G\"odel modal
logics over witnessed models not only clarifies their semantic
foundations but also strengthens their connections with well-behaved
fragments of first-order logic that play a central role in computer
science, thereby enhancing both their theoretical robustness and their
applicability.

%%By eliminating pathological behaviors arising from unattained infima 
%% and suprema, it avoids limit-based counterexamples and yields a 
%%more constructive semantics. 
%%This tight correspondence between semantic witnesses and syntactic
%%derivations naturally
From a proof-theoretic perspective, witnessed semantics aligns more
closely with syntactic calculi.
%% and this suggests the development of a
%% complete proof system. 
The existence of witnesses enables canonical model constructions that
are finite or finitely generated, supports filtration techniques, and
allows completeness proofs to proceed by standard algebraic or
canonical methods. Therefore, establishing a sound and complete
calculus for G\"odel modal logics over witnessed models is not only
feasible but also conceptually justified, providing a proof-theoretic
counterpart to their improved semantic behavior and paving the way for
effective automated reasoning procedures.


In line with the above considerations, in~\cite{FerFioRod:2025} we
introduced the logic $\GWCL$ characterized by witnessed crisp G\"odel
Kripke models, where ``crisp'' means that the accessibility relation
takes values in the set $\{0,1\}$.  For this logic, we developed a
constraint-based refutation calculus inspired by the tableau system
of~\cite{BilkovaFK:22}, together with a terminating proof-search
procedure that yields finite countermodels whenever proof search
fails. As a consequence, we established that $\GWCL$ enjoys the finite
model property.  In the present work, we extend these results to the
logic $\GWL$, characterized by witnessed G\"odel–Kripke models without
the crispness restriction. Our development follows the
strategy adopted in~\cite{FerFioRod:2025}.  We introduce a
constraint-based refutation calculus for $\GWL$, we call $\Calcw$, and
prove its soundness and completeness with respect to witnessed
models. Moreover, we design a terminating proof-search procedure for
$\Calcw$ based on a standard backward search strategy.  Notably, when
proof search fails, the computation can be traced to construct a
finite countermodel for the given formula, and this proves the finite
model property for $\GWL$.  Finally, we provide an implementation of
the procedure, called \texttt{gwref} (available
at~\cite{gwrefProver}).




%%% Local Variables: 
%%% mode: latex
%%% TeX-master: "goedelModalLogicWitnessNonCrisp"
%%% End: 

\section{Basic Definitions}
\label{sec:basic}

Formulas are built over a countable set of propositional variables
$\PV$, using the propositional connectives $\land$, $\lor$, $\to$, the
constant $\bot$ and the modal connectives $\Box$, $\Diam$.  We write
$\neg \varphi$ as a shorthand for $\varphi\to \bot$.  The symbols
$\land$, $\lor$ and $\to$ are also used to denote algebraic
operations.
The \emph{G\"odel algebra} is a structure
$\stru{[0,1],\land,\lor,\to}$ where:
\[
a\land b = \min(a,b)
\qquad
a \lor b = \max(a,b)
\qquad
a\to b =
\begin{cases}
  1 & \mbox{if $a\leq b$}\\
  b &  \mbox{if $a > b$}     
\end{cases}
\]

%% Gc-model
\noindent
 A \emph{\GM-model} (\emph{G\"odel  model}) $\M$ is a structure
 $\stru{W,R,e}$ where $W$ is a nonempty set (the set of worlds),
$R$ is a map $W\times W\to [0,1]$ and  $e : W \times \PV\to [0,1]$.
If $R : W\times W\to \{0,1\}$, we say that $\M$ is \emph{crisp};
in this case,  $R \subseteq W\times W$. 
%We write $w_1 R w_2$ to mean that $(w_1,w_2)\in R$;
The map $e$ is extended to arbitrary formulas as follows:
\begin{itemize}
\item $e(w,\bot) = 0$;
\item $e(w,\a \star \b) = e(w,\a) \star e(w,\b)$, for $\star\in \{\,\land,\,\lor,\,\to\,\}$;
\item $e(w,\Box \a) = \inf_{w'\in W}  \{\,R(w,w') \to e(w',\a)  \,\}$;
\item $e(w,\Diam \a) =  \sup_{w'\in W}  \{\,R(w,w') \land e(w',\a)  \,\}$.
\end{itemize}


\noindent
Note that:
%$e(w,\neg\a)\in \{0,1\}$, and $e(w,\neg\a)=1$ iff $e(w,\a)=0$. 
\[
 e(w,\neg\a) =
 \begin{cases}
   1 & \mbox{if $e(w,\a)=0$}    
   \\
   0 &   \mbox{if $e(w,\a)> 0$} 
 \end{cases}
 \qquad
 e(w,\neg\neg\a) =
 \begin{cases}
   1 & \mbox{if $e(w,\a) > 0$}    
   \\
   0 &   \mbox{if $e(w,\a)= 0$} 
 \end{cases}
\]
\noindent
A \GM-model $\stru{W,R,e}$ is
\emph{witnessed} iff every world $w$ having at least one $R$-successor
satisfies the following properties:
\begin{itemize}
\item  $e(w,\Box \a)=r$ iff 
   there is $w'\in W$   such that  $R(w,w') \to e(w',\a) = r$;

\item if $e(w,\Diam \a)=r$ iff
   there is $w'\in W$  such that $R(w,w')\land e(w'\a) = r$.
  
\end{itemize}
Thus, the values   $e(w,\Box\a)$  and $e(w,\Diam\a)$  are witnessed by  $w'$.
As a consequence, in the definitions of $e(w,\Box \a)$ and  $e(w,\Diam \a)$,
the infimum and the superior are just the minimum and the maximum.
Note that  every finite \GM-model is witnessed.
A world $w'$ is an $R$-successor of $w$
iff  $R(w, w')>0$. If  $w_0$ has no $R$-successors,
then $e(w_0,\Box\varphi)=1$ and $e(w_0,\Diam\varphi)=0$;
indeed,   for every $w'\in W$,
$R(w_0,w')=0$,
hence $R(w_0,w') \to e(w',\varphi) = 1$ and $R(w_0,w')\land e(w'\varphi) = 0$.
A formula $\varphi$ is \emph{valid} in $\M=\stru{W,R,e}$ iff
$e(w,\varphi)=1$ for every world $w$ in $W$.  If $\varphi$ is not
valid in $\M$, we say that $\M$ is a \emph{countermodel} for
$\varphi$; thus, $\M$ contains a world $w$ such that $e(w,\varphi) <
1$.
A model $\M=\stru{W,R,e}$ is \emph{discrete} if $W$ is
finite and the images of $R$ and  $e$ are finite subsets of $\Qrange$, the set
of rational numbers in the range $[0,1]$.




Let  $\GL$ be the G\"odel-Dummett Logic, obtained by by extending
Intuitionistic Propositional Logic $\IPL$ with the linearity axiom
$(\a\to\b)\lor (\b\to \a)$.
%We recall the axiomatization of $\GCL$ given in~\cite{RodriguezV:21}.
G\"odel Modal Logic $\GKL$ is obtained by
adding to $\GL$ the following axioms and rules ($\vdash$ refers to
provability in $\GKL$):
\[
\begin{tabular}{m{3em}m{12em}m{3em}m{16em}}
  $(K_\Box)$ & $\Box(\a \to \b)\to (\Box\a\to \Box \b)$ & 
  $(K_\Diam)$ & $\Diam(\a \lor \b)\to (\Diam\a\lor \Diam \b)$
  \qquad $(F_\Diam)\; \neg \Diam\bot$ 
  \\
  $(FS_1)$ & $\Diam(\a \to \b)\to (\Box\a\to \Diam \b)$ & 
  $(FS_2)$ & $(\Diam \a \to\Box \b)\to \Box(\a\to  \b)$
  \\
  $(N_\Box)$ & $\vdash \a$ implies $\vdash \Box\a$ &
  $(N_\Diam)$ & $\vdash \a\to \b$ implies $\vdash \Diam \a\to\Diam\b$                                                   
\end{tabular}
\]
We introduce some logics extending   $\GKL$, by providing a semantic characterization;
the mutual relationships between these logics  are displayed in Fig.~\ref{fig:diagram}
and discussed below.

\begin{itemize}
\item $\GCL$ is the set of formulas valid in every crisp \GM-model (see~\cite{RodriguezV:21});
\item  $\GWL$  is the set of formulas valid in every witnessed \GM-model;
\item $\GWCL$ is the set of formulas valid in every  witnessed crisp \GM-model (see~\cite{FerFioRod:2025}).
\end{itemize}


\noindent
The logic $\GCL$ is obtained by adding the axiom $\Box(\a\lor\b)\to (\Box \a \lor \Diam \b)$
to  $\GKL$ (see~\cite{RodriguezV:21}).
 The logic $\GWCL$ has been introduced in~\cite{FerFioRod:2025} and no axiomatization is known.
All the inclusions between the logics shown in Fig.~\ref{fig:diagram}  are strict.
Let $\psi=\Box \neg\neg \a\to \neg\neg \Box \a$. 
In~\cite{FerFioRod:2025},
it is proved that   $\psi\in\GWCL$ and $\psi\not\in\GCL$.
We show that $\psi\in\GWL$.
Let $\M=\stru{W,R,e}$ be a witnessed $\GM$-model and $w\in W$.
If $e(w, \neg\neg \Box \a)=1$, we immediately get $e(w,\psi)=1$. Otherwise,
it holds that $e(w, \neg\neg \Box \a)=0$, hence $e(w, \Box \a)=0$.
Accordingly,  there exists a witnessing $w'\in W$ such that $R(w,w') \to e(w',\a) = 0$;
it follows that   $e(w',\a)=0$, hence  $e(w',\neg \neg \a)=0$.
This implies  $e(w,  \Box\neg \neg  \a)=0$,
hence $e(w, \psi) = 1$. Having proved that $e(w, \psi) = 1$, we conclude  $\psi\in \GWL$.


Let $\varphi =\Box (p\lor q)\to (\Box p \lor \Diam q)$;
we show that $\varphi\not\in\GWL$
by providing a  countermodel for $\varphi$.
Let $\M=\stru{W,R,e}$ be the  witnessed  $\GM$-model in Fig.~\ref{fig:countGW};
in the figure we only display the values of $R$ and $e$ different from zero.
Since $e(w_0,\varphi) = 0.6$, $\M$ is a countermodel for $\varphi$, and this certifies that
$\varphi\not\in\GWL$.


We point out that, in all the mentioned modal logics, $\Box$ and $\Diam$ are not
interdefinable.
This is proved for  $\GCL$ in~\cite{RodriguezV:21};
in App.~\ref{app:noninterdBD} the proof is extended to all the other  modal logics mentioned above.





\begin{figure}[t]
  \centering
  

\begin{minipage}{\linewidth}
  \begin{center}
  \begin{tikzpicture}[
    level 1/.style={sibling distance=10em},
    % level 2/.style={sibling distance=4em},
    ]
    \node[fill=green!25] (GWc) {$\GWCL$}
    child{ 
      node[fill=green!25] (Gc) {$\GCL$}
      child[missing]{}
      child{
        node (GK)[fill=green!25]  {$\GKL$}
        edge from parent
        %node[right=.5ex] {$\subset$}
      } 
      edge from parent
      %node[right=.5ex] {$\subset$}
    }
    child{ 
      node[fill=green!25] (GW) {$\GWL$}
      edge from parent
      %node[left=.5ex] {$\subset$}
    } 
    ;

    

    %\draw (GW) -- node[left=.5ex] {$\subset$} (GK) ;    
    \draw (GW) -- (GK) ;    
    

    \node[right=1ex of GWc] {\begin{minipage}{16ex}
      \small (1), (2) valid\end{minipage}};
    \node[right=2ex of GW] {\begin{minipage}{16ex}
      \small (1) valid\par (2) NOT valid\end{minipage}};
    \node[left=-2ex of Gc] {\begin{minipage}{16ex}
      \small (1) NOT valid\par (2)  valid\end{minipage}};
    \node[right=1ex of GK] {\begin{minipage}{20ex}
      \small (1), (2) NOT valid\end{minipage}};

  \end{tikzpicture}
  \begin{minipage}{60ex}
    \begin{tabular}{p{30ex}p{50ex}}
      \begin{enumerate}[label=(\arabic*)]
      \item $\Box \neg\neg \a \to \neg\neg\Box \a$
      \item $\Box (\a \lor \b)\to (\Box \a \lor \Diamond \b)$
      \end{enumerate}
      &
      \begin{itemize}
      \item $\GCL = \GKL + (2)$ (see \cite{RodriguezV:21})
      \item $\GKL\subset\GCL\subset\GWCL$,   $\GKL\subset\GWL\subset\GWCL$
      \end{itemize}
    \end{tabular}
  \end{minipage}
\end{center}
\end{minipage}

%%% Local Variables: 
%%% mode: latex
%%% TeX-master: "goedelModalLogicWitnessNonCrisp"
%%% End: 

  %
\begin{verbatim}
     ---  GW^c ---  (1),(2) valid                  
   /              \
  /                \
 /                  \
G^c  (1) NOT valid      GW  (1) valid              
     (2) valid              (2) NOT valid
  \                   /
   \                 /
    \               /
      ---  GK   ---
           (1), (2) NOT valid   


G^c = GK + (2) [Rodriguez&Vidal,SL2021]

(1)  Box ~~A -> ~~ Box A
(2)  Box(A v B) -> (Box A v Diam B)
\end{verbatim}





% %%% Local Variables: 
%%% mode: latex
%%% TeX-master:  "goedelModalLogicWitnessNonCrisp"
%%% End: 

  \caption{Logics overview.}
  \label{fig:diagram}
\end{figure}


\begin{figure}[t]
  \centering
\[\small
  \begin{array}{c}
    \begin{minipage}{10em}
    \begin{tikzpicture}[scale=0.8]
      % WORLDS  
      % w0
      \draw[fill] (0,0) circle (3pt)
      +(0,-0)   node (w0)  {}  % used to label the circle 
      +(0,-0.3) node  {$w_0$}
      ;
      
      % w1 
      \draw[fill]  (0,1.5) circle (3pt)
      +(0,0)   node (w1)  {}  % used to label the circle 
      +(-0.6,0)  node  {$w_1$} 
      +(1.2,0.2) node  {$p=0.5$}
      +(1.2,-0.2) node  {$q=0.6$}
      ;
      

      % ARROWS
      \draw[->] (w0) -- (w1) node[above left=-1.4] {0.6} ;
    \end{tikzpicture}
  \end{minipage}
  \begin{minipage}{20em}
    \[\small
    \begin{array}{l}
      % -----------------------------------------      
      \fcolorbox{black}{gray!10}{
        \begin{minipage}{15em}\small
          $W=\{\,w_0,\,\,w_1\,\}$
          \\[.5ex]
          $R(w_0,w_1) = 0.6$
          \\[.5ex]
          $e(w_1,p) = 0.5,\; e(w_1,q)=0.6$
        \end{minipage}
      } % end box    
      % -----------------------------------------  
    \end{array}
\]
  \end{minipage} 
\\[10ex]
    \begin{array}{rcl}
      e(w_0,\Box (p\lor q))&=&\inf\{\, R(w_0,w_0)\to e(w_0,p\lor q),\, R(w_0,w_1)\to e(w_1,p\lor q) \,\}\,=\,
\inf\{\, 0\to 0,\, 0.6\to 0.6 \,\}\,=\,1
    \\[1ex]
    e(w_0,\Box p) &=& \inf\{\, R(w_0,w_0)\to e(w_0,p),\   R(w_0,w_1)\to e(w_1,p)\,\} \,=\,

              \inf\{\, 0\to 0,\  0.6\to 0.5\,\}    \,=\, 0.5
\\[1ex]
    e(w_0,\Diam q)&=& \sup\{\, R(w_0,w_0)\land e(w_0,q),\, R(w_0,w_1)\land e(w_1,q)\,\}\,=\,
\sup\{\, 0\land 0,\,  0.6\land 0.6\,\}\,=\,  0.6
\\[1ex]
      e(w_0,\varphi) &=& e(w_0, \Box (p\lor q))\to (e(w_0, \Box p) \lor e(w_0,\Diam q)) \,=\, 1\to  (0.5 \lor 0.6) \,=\, 0.6
\end{array}
    \end{array}
  \]
  
  \caption{A countermodel  for the formula $\varphi=\Box (p\lor q)\to (\Box p \lor \Diam q) $.}
  \label{fig:countGW}
\end{figure}


\noindent






%%% Local Variables: 
%%% mode: latex
%%% TeX-master: "goedelModalLogicWitnessNonCrisp"
%%% End: 

\section{The Calculus $\Calcw$}
\label{sec_calculus}

%%#############################  CALCULUS - PROPOSITIONAL RULES  ################################
\begin{figure}[t]
  \[%\large
  \centering
  \begin{array}{c}
    %% AXAT 
    \mbox{$\lt\in\{\,<,\,\leq\,\}$,\hspace{1em} $\gt \in \{\,>,\,\geq\,\}$}\qquad\qquad
    \AXC{}
    \RightLabel{\small$\ruleAxat$}
    \UIC{$\G$}    
    \DP
    \quad\mbox{if $\Atmp{\G}$ is not consistent}
    \\[4ex]
    %% AND GT 
    \AXC{$w: \a \gt t,\, w: \b\gt t,\,  \G$}    
    % ------------------------------------------    
    \RightLabel{$\land\gt$}
    \UIC{$w: \a\land \b \gt t,\,\G$}
    \DP
    \hspace{4em}
    %% AND LT 
    \AXC{$w: \a \lt t,\, \G$}    
    \AXC{$w: \b \lt t,\, \G$}
    % ------------------------------------------      
    \RightLabel{$\land\lt$}
    \BIC{$w: \a\land \b \lt t,\,\G$}
    \DP
    \\[4ex]    
    %%%%%%%%%%%%%%%%%%%%%%%%%%%%%%%%%%%%%%%%%%%%%%%%%%%%%%%% 
    %% OR LT 
    \AXC{$w: \a \lt t,\, w: \b \lt t,\,  \G$}    
    % ------------------------------------------    
    \RightLabel{$\lor\lt$}
    \UIC{$w: \a\lor \b \lt t,\,\G$}
    \DP
    \hspace{4em}
    %% OR GT 
    \AXC{$w: \a \gt t,\,  \G$}
    \AXC{$w: \b \gt t,\,  \G$}
    % ------------------------------------------        
    \RightLabel{$\lor\gt$}
    \BIC{$w: \a\lor \b \gt t,\,\G$}
    \DP
 \\[4ex]
    %%%%%%%%%%%%%%%%%%%%%%%%%%%%%%%%%%%%%%%%%%%%%%%%%%%%%%%%%%%%%%%%%%%%%%
    %% IMP LTS prop
 \AXC{$w :\a >w : p,\, w : p  < t,\,  \G$}    
    % ------------------------------------------    
    \RightLabel{$\to < \;(\dag)$}
    \UIC{$w: \a\to p < t,\,\G$}
    \DP
\hspace{1em}
    %% IMP LTS 
    \AXC{$w :\a > w : q,\,  w :\b \leq w :q,\,  w : q  < t,\,  \G$}    
    % ------------------------------------------    
    \RightLabel{$\to < \;(\dag)$}
    \UIC{$w: \a\to \b < t,\,\G$}
    \DP
    \\[4ex]
    %% IMP LTE prop
    \AXC{$t \geq 1,\, \G$}    
    \AXC{$w :\a >w : p,\,w : p \leq  t,\,\G$}    
    % ------------------------------------------    
   % \insertBetweenHyps{\hskip -0.01em}
    \RightLabel{$\to\leq\,(\dag)$}
    \BIC{$w: \a\to p \leq t,\,\G$}
    \DP
\\[4ex]    
    %% IMP LTE
    \AXC{$t \geq 1,\, \G$}    
    \AXC{$  w :\a > w : q,\,    w :\b \leq w : q,\,   w : q \leq  t,\,\G$}    
    % ------------------------------------------    
    \RightLabel{$\to\leq\,(\dag)$}
    \BIC{$w: \a\to \b \leq t,\,\G$}
    \DP
    \\[4ex]   
 %% IMP GT prop
    \AXC{$w :\a \leq w : p,\,1\gt t,\, \G$}    
    \AXC{$w:p \gt t,\, \G$}
    % ------------------------------------------       
    \RightLabel{$\to\gt\,(\dag)$}
    \BIC{$w: \a\to p \gt t,\,\G$}
    \DP
    \\[4ex]
    %% IMP GT 
    \AXC{$w :\a\ \leq w : q,\, w : \b \geq w :q ,\, 1\gt t,\, \G$}    
    \AXC{$w:\b \gt t,\, \G$}
    % ------------------------------------------       
    \RightLabel{$\to\gt\,(\dag)$}
    \BIC{$w: \a\to \b \gt t,\,\G$}
    \DP
    %%% WHERE ....
    \\[2ex]
    \begin{minipage}{1.0\linewidth}
      \[
      \begin{array}{l}
        \Atmp{\G} \;=\;
        \Atm{\G}\;\cup\;
        \{\, 1 > t~|~ w : \Box \a > t \in \G\,\}\;\cup\;
        \{\, 0 < t~|~ w : \Diam \a < t \in \G\,\}
        \\[1ex]   
        (\dag)\quad p\in\PV\cup\{\bot\},\;  \b\not\in\PV\cup\{\bot\},\;\mbox{$q$ is a new propositional variable}
        \end{array}
       \]
    \end{minipage}   
  \end{array}
  \]
 \caption{The calculus $\Calcw$, propositional rules ($\Gamma$ is a multiset of constraints).}
  \label{fig:calc1}
\end{figure}
%################



%%#############################  CALCULUS - MODAL RULES  ################################
\begin{figure}[t]
  \[%\large
  \centering
  \begin{array}{c}
    %% BOX LT        
     \AXC{$R(w,w_1)\to w_1:\a \lt t,\;\Phibd{\G,w,w_1},\;\G$}
     % -----------------------------------------
     \RightLabel{$\Box\lt$}
     \UIC{$w:\Box\a \lt t,\, \G$}    
     \DP
     \hspace{3em}
     %% DIAM GT L     
     \AXC{$R(w,w_1)\land w_1:\a \gt t,\;\Phibd{\G,w,w_1},\;  \G$}
     % ------------------------------------------
     \RightLabel{$\Diam\gt$}
     \UIC{$w:\Diam\a \gt t,\,\G$}    
    \DP
     %%% WHERE ....
     \\[2ex]
     \begin{minipage}{1.0\linewidth}
       \[
       \begin{array}{l}
         \mbox{$w_1$ is a new label}
        \\[1ex]
         \Phibd{\G,w, w_1}\,=\,
         \{\, R(w,w_1)\to w_1: \b \gt t~|~w :\Box \b\gt t \in\G  \,\}\,\cup\,
         \{\, R(w,w_1)\land w_1: \b \lt t~|~w :\Diam \b\lt t \in\G  \,\}
        \end{array}
\]
\end{minipage}
\end{array}
    \]
    \caption{The calculus $\Calcw$, modal rules ($\Gamma$ is a multiset of constraints).}
  \label{fig:calc2}
\end{figure}
%################



%%#############################  CALCULUS - R-RULES  ################################
\begin{figure}[t]
  \[%\large
  \centering
  \begin{array}{c}
    %% R AND GT 
    \AXC{$R(w,w') \gt t,\, w': \a\gt t,\,  \G$}    
    % ------------------------------------------    
    \RightLabel{$R\land\gt$}
    \UIC{$R(w,w') \land w': \a \gt t,\,\G$}
    \DP
    \hspace{4em}
    %% R AND LT 
    \AXC{$R(w,w') \lt t,\, \G$}    
    \AXC{$w': \a \lt t,\, \G$}
    % ------------------------------------------      
    \RightLabel{$R\land\lt$}
    \BIC{$R(w,w') \land w':\a \lt t,\,\G$}
    \DP
    \\[4ex]
%==============================================================
    %% R IMP LTS - PROP
    \AXC{$R(w,w')> w': p,\,  w':p  < t,\,  \G$}    
    % ------------------------------------------    
    \RightLabel{$R\to <\,(\dag)$}
    \UIC{$R(w,w') \to w':p < t,\,\G$}
    \DP
\\[4ex]
 %% R IMP LTS 
    \AXC{$R(w,w')> w': q,\,w':\a \leq w':q,\,  w':q  < t,\,  \G$}    
    % ------------------------------------------    
    \RightLabel{$R\to <\,(\dag)$}
    \UIC{$R(w,w') \to w':\a < t,\,\G$}
    \DP
    \\[4ex]    
    %% R IMP LTE - PROP
    \AXC{$t \geq 1,\, \G$}    
    \AXC{$R(w,w')> w':p,\,  w':p \leq  t,\,\G$}    
    % ------------------------------------------    
    \RightLabel{$R\to\leq\,(\dag)$}
    \BIC{$R(w,w')\to w':p \leq t,\,\G$}
    \DP
\\[4ex]    
    %% R IMP LTE 
    \AXC{$t \geq 1,\, \G$}    
    \AXC{$R(w,w')> w':q,\,w':\a\leq w':q,\,  w':q \leq  t,\,\G$}    
    % ------------------------------------------    
    \RightLabel{$R\to\leq\,(\dag)$}
    \BIC{$R(w,w')\to w':\a \leq t,\,\G$}
    \DP
    \\[4ex]   
    %% R IMP GT 
    \AXC{$w' :\a \geq R(w,w'),\,1\gt t,\, \G$}    
    \AXC{$w':\a \gt t,\, \G$}
    % ------------------------------------------       
    \RightLabel{$R\to\gt$}
    \BIC{$R(w,w')\to w':\a \gt t,\,\G$}
    \DP
%%% WHERE ....
    \\[4ex]
     (\dag)\quad p\in\PV\cup\{\bot\},\;  \a\not\in\PV\cup\{\bot\},\;\mbox{$q$ is a new propositional variable}
  \end{array}
  \]
  \caption{The calculus $\Calcw$,  $R$-rules ($\Gamma$ is a multiset of constraints).}
  \label{fig:calc3}
\end{figure}
%#########################################################################





We introduce the calculus $\Calcw$ for $\GWL$;
we call  \GWM-model  a witnessed \GM-model.
We mainly
follow~\cite{BilkovaFK:22,FerFioRod:2025}.
% We mainly
% follow~\cite{BilkovaFK:22}, where the calculus $\TKGL$ for $\KGL$ is
% presented.  The logic $\KGL$ is an extension of $\GWCL$ over a more
% expressive language, including an involutive negation and a
% co-implication. Note that $\KGL$ has the finite model property, hence
% $\KGL$-models are witnessed.
The constraint language $\LC$ is defined over a countable set of 
labels, each representing a world of a \GWM-model. 
In the following definitions $w$, $w'$ are labels of $\LC$, $\varphi$ is a
formula,  $p\in \PV$,  $r\in\Qrange$,   $R$ is a designated relation symbol,
representing the accessibility relation, and
$\abstractorder\in \{<,\leq,>,\geq\,\}$.  
\[
\begin{array}{rcl}
  \mbox{labelled formula} & \;\coloneqq&\; w :\varphi                            
                             \\
  \mbox{atomic c-term $t$}  &\; \coloneqq&\; r~|~w :p~|~w :\bot~|~R(w,w')
  \\[.5ex]
  \mbox{c-term $u$}  & \;\coloneqq&\; t~|~w : \varphi~|~R(w,w')\land w':\varphi~|~R(w,w')\to w':\varphi
 \\[.5ex]
  \mbox{constraint $\chi$}   &\; \coloneqq&\; u \abstractorder t
\end{array}
\]
If  $w:\varphi$ occurs in   a c-term $u$, we say  that $u$ has label $w$.
A constraint of the form $t \abstractorder t$ is called \emph{atomic};
if $\chi= w:\varphi \abstractorder t$ is non-atomic and $\sharp$ is
the main connective of $\varphi$, we say that $\chi$ is a
\emph{$\sharp$-constraint}.
Constraints where $u$ is   $R(w,w')\land w':\varphi$ or
$R(w,w')\to w':\varphi$ are called
\emph{$R$-constraints}.
Given a multiset of constraints $\G$, by
$\Atm{\G}$ we denote the set of atomic constraints in $\G$.
Let $\M=\stru{W,R,e}$ be a \GWM-model. An $\M$-interpretation of
$\LC$ is a function $\Ical$ mapping labels of $\LC$ to $W$.
We extend $\Ical$ to c-terms as follows:
\begin{itemize}
\item $\Ical(r) = r$, for every $r\in \Qrange$;
\item $\Ical( w:\varphi) = e(\Ical(w), \varphi)$;
\item $\Ical\left(R(w,w')\right) =  R\left(\Ical(w),\Ical(w')\right)$;
\item $\Ical\left(R(w,w') \odot w':\varphi\right) = R\left(\Ical(w),\Ical(w')\right) \,\odot\,   e\left(\Ical(w'), \varphi\right)$,
  where $\odot\in\{\land, \to\}$.
 \end{itemize}
Note that, for every c-term $u$, $\Ical(u)$ belongs to $[0,1]$.
 We introduce the relations
$\models_\Ical$ and $\models$, where $\M$ is a \GWM-model, $\Ical$ an
$\M$-interpretation, $\Gamma$ a multiset of constraints.
\begin{itemize}
\item $\M\models_\Ical u \abstractorder t$ iff   $\Ical(u)  \abstractorder \Ical(t)$;
  
\item $\M\models_\Ical \Gamma$ iff $\M\models_\Ical\chi$, for every
  $\chi\in\Gamma$;

\item $\M\models \Gamma$ iff $\M\models_\Ical \Gamma$ for some
  $\M$-interpretation $\Ical$.
\end{itemize}
A \emph{substitution} $\s$ is a function mapping each atomic c-term of the
form  $w : p$  or $R(w,w')$ to a rational number in $\Qrange$; $\s$
is extended to all the atomic c-terms by setting $\s(r)=r$, for $r\in
\Qrange$, and $\s(w:\bot)=0$.  Let $\Gat$ be a set of atomic
constraints.  By $\s(\Gat)$ we denote the set of constraints obtained
by replacing every atomic c-term $t$ occurring in $\Gat$ with $\s(t)$;
note that $\s(\Gat)$ is a set of rational constraints of the form $r_1
\abstractorder r_2$, with $r_1$ and $r_2$ in $\Qrange$.
Consistency of a set of atomic constraints is defined as follows:
\begin{itemize}[leftmargin=*]
\item $\Gat$ is \emph{consistent} iff there exists a substitution $\s$
  such that all the constraints in $\s(\Gat)$ hold; we call $\s$ a
  \emph{solution} to $\Gat$.
\end{itemize}
We remark that consistency of $\Gat$ can be checked by a Constraint
Solver over $Q$: one has to abstract  the c-terms $w : p$ and $R(w,w')$ occurring in $\Gat$
by introducing new variables ranging over
$\Qrange$, and then check the consistency of the obtained constraints
by exploiting the solver.



Hereafter, $\lt\in\{<,\leq\}$ and $\gt \in \{>,\geq\}$. The rules of the calculus
$\Calcw$ are displayed in Fig.~\ref{fig:calc1} (propositional rules),  Fig.~\ref{fig:calc2}
(modal rules) and  Fig.~\ref{fig:calc3} ($R$-rules).
%%The axiom rules (namely, the rules without premises) are $\ruleAxat$,
%%$\ruleAxz$ and $\ruleAxo$.  
% It consists of propositional rules, i.e., the axiom rule $\ruleAxat$
% and rules for the propositional connectives,  the modal rules
% $\Box\lt$ and $\Diam\gt$ and rules to handle $R$-constraints.
The \emph{main constraint} of a rule application is the constraint displayed in the conclusion.
Rules for implication, having a main constraint of the kind
$w:\a\to \b\abstractorder t$, are defined according to the structure of $\b$.
Let us consider the rule $\to <$, having main
constraint $w:\a\to \b < t$.
If $\b=p$, with $p\in \PV\cup\{\bot\}$, the premise contains the constraints $w:\a > w:p$ and $w:p < t$.
This reflects the fact that, given a model $\M$, a world $w$ in $\M$ and an  $\M$-interpretation
$\Ical$, if $e(\Ical(w), \a\to p) < \Ical(t)$, then $e(\Ical(w),\a) > e(\Ical(w),p)$,
 $e(\Ical(w), \a\to p) =  e(\Ical(w),p)$
and  $e(\Ical(w),p) <  \Ical(t)$.
This  reasoning cannot be generalized to any $\b$ since
the constraint  $w:\a > w:\b$ is not allowed if  $\b\not\in\PV\cup\{\bot\}$.
In the latter case,
we introduce a new propositional variable $q$, behaving in $w$  as $\b$, and
$w:\a > w:\b$ is replaced by $w:\a > w: q$.
The correspondence between $\b$ and $q$ is set by the
 constraint $w:\b\leq w:q$,
while the converse constraint $w:\b\geq w:q$ can be omitted.
There are two modal rules (see Fig.~\ref{fig:calc2}), the former   having  main constraint  $w:\Box\a \lt t$
and the latter $w:\Diam\a \gt t$;
both rules introduce a new label $w_1$.
 $R$-rules (see Fig.~\ref{fig:calc3})
are similar to the corresponding propositional rules.
The definitions of $\Calcw$-tree and
$\Calcw$-derivation are the usual ones (see, e.g.,~\cite{TroSch:00}).

\inlinetodo{
We remark that $\Calcw$ has the \emph{subformula property}, namely: if
$\Tcal$ is a $\Calcw$-tree having $\G_0$ as root, every formula
occurring in $\Tcal$ inside a constraint  (e.g., the formula $\varphi$
in the constraint $R(w,w')\to w':\varphi$) is a subformula of a formula in $\G_0$
or a new propositional variable.
}
By $\provesw{\Gamma}$ we mean that there exists a
$\Calcw$-derivation of $\Gamma$.  In the rest of this section we show
that $\Calcw$ is a sound and complete calculus for $\GWL$, where
soundness and completeness are formalized as follows:

\begin{itemize}
\item (Soundness) if $\provesw{\Gamma}$, then $\M\not\models\G$, for every  \GWM-model $\M$;
\item (Completeness) if $\M\not\models\G$, for every \GWM-model  $\M$, then $\provesw{\Gamma}$.
\end{itemize}
We stress that in rule $\ruleAxat$ it is not enough to check the
consistency of $\Atm{\G}$.
Indeed, let us consider the set of constraints $\G=\{w : \Box p > w:p,\,w:p\leq 1,\,w:p\geq 1\}$.
Clearly, there is no  \GWM-model   $\M$  such that
$\M\models \G$ hence,  by completeness,  $\G$ must  be  provable in $\Calcw$.
The only possible way to prove $\G$ is by an application of rule $\ruleAxat$.
If $\ruleAxat$  checked the consistency of  $\Atm{\G}$, 
 $\G$ would not be  proved, since  $\Atm{\G}=\{w:p\leq 1,\,w:p\geq 1 \}$  is  consistent.
Instead,  $\ruleAxat$ evaluates  $\Atmp{\G} =\Atm{\G}\cup \{1 > w:p\}$;
since $\Atmp{\G}$ is not consistent,  $\G$ is proved by $\ruleAxat$.
% However, there is no  \GWM-model   $\M=\stru{W,R,e}$  and $\Ical$ such
% that $\M\models_\Ical \G$, since $e(\Ical(w), \Box \a) \leq 1$.
% In $\Calcw$,  $\G$ is provable in $\Calcw$ since rule
% $\ruleAxat$ checks the consistency of $\Atmp{\G} = \{1 > 1\}$.
% Indeed, let $\G=\{w : \Box \a >w:p, w:p \leq 1,
% w:p \geq 1\}$. Clearly, there are no  \GWM-model  $\M=\stru{W,R,e}$  and $\Ical$ such
% that $\M\models_\Ical \G$, otherwise we would get $e(\Ical(w),p)=1$ and
% $e(\Ical(w), \Box \a) >  e(\Ical(w),p)$, namely $e(\Ical(w), \Box \a) > 1$,
% a contradiction.  We point out that
% $\Atm{\G}$ is consistent, but $\G$ is provable in $\Calcw$ since rule
% $\ruleAxat$ checks the consistency of
% $\Atmp{\G} = \{1 > w:p,\,w:p \leq 1,\,w:p \geq 1\}$.






\bigskip 
\inlinetodo{**TO DO**
An example of $\Calcw$-derivation is shown in Fig.~\ref{fig:der}.  
In representing derivations, we underline the main constraint of a rule
application; moreover, we omit redundant constraints (e.g.,
constraints of the kind $t \leq t$ and multiple copies of the same
constraint).}


To prove the soundness of $\Calcw$, we need the
following:

%%######### EXAMPLE DERIVATION #########
\begin{figure}[t]
  \centering\small
  ** TO DO **
  % \[
  % \begin{array}{l}
  %   \AXC{}
  %   \RightLabel{$\ruleAxat$}
  %   \UIC{$\G_1$}
  %   % ////////////////////////////////////////////////////////////////////////////////////////
  %   \AXC{}
  %   \RightLabel{$\ruleAxat$}
  %   \UIC{$\G_3$}
  %   % //////////////////
  %   \AXC{}
  %   \RightLabel{$\ruleAxat$}
  %   \UIC{$\G_5$}
  %   % ////////
  %   \AXC{}
  %   \RightLabel{$\ruleAxat$}
  %   \UIC{$\G_6$ }
  %   % ==========================
  %   \RightLabel{$\to \leq$}
  %   \BIC{$\underline{w_1:\neg p \leq c_2},\,w_1:\bot \geq c_2,\, 1 > c_0,\,  w_1:p \leq c_1,\,\D$}
  %   % /////
  %   \AXC{}
  %   \RightLabel{$\ruleAxat$}
  %   \UIC{$\G_4$}
  %   % ===================
  %   \RightLabel{$\to \gt$}
  %   \BIC{$w_1:p \leq c_1,\,\underline{w_1:\neg \neg p > c_0},\,\D$}
  %   % //////////////////
  %   \RightLabel{$\Box\lt$}
  %   \BIC{$\underline{w: \Box p \leq c_1},\,\D$}
  %   % /////////////////////////////////////
  %   \AXC{}
  %   \RightLabel{$\ruleAxat$}
  %   \UIC{$\G_2$}
  %   % ==================================
  %   \insertBetweenHyps{\hskip -5em}
  %   \RightLabel{$\to \gt$}
  %   \BIC{$\underline{w:\neg  \Box p > w:\bot},\,w:\bot \leq c_0,\,w: \Box \neg \neg p > c_0,\,c_0 < 1$ }
  %   % ==========================================================================================
  %   \RightLabel{$\to \leq$}
  %   \insertBetweenHyps{\hskip -7em}
  %   \BIC{$\underline{w:\neg \neg  \Box p \leq c_0},\;w: \Box \neg \neg p > c_0,\;c_0 < 1$}
  %   % -----------------------------------------------------
  %   \RightLabel{$\to <$}
  %   \UIC{$\underline{w: \Box \neg \neg p \to \neg \neg  \Box p < 1}$}
  %   \DP
  %   % #############################################    
  %   \\[14ex]
  %   \D = w:\bot \geq c_1,\,1 > w:\bot,\,w:\bot \leq c_0,\,
  %   w: \Box \neg \neg p > c_0,\,c_0 < 1
  %   \\[.5ex]    
  %   \G_1=c_0 \geq 1,\,   w: \Box \neg \neg p > c_0,\,c_0 < 1
  %   \\[.5ex]    
  %   \G_2 = w:\bot > w:\bot,\,w:\bot \leq c_0,\, w: \Box \neg \neg p > c_0,\,c_0 < 1
  %   \\[.5ex]    
  %   \G_3 =1 \leq c_1,  w:\bot \geq c_1,\,1 > w:\bot,\,w:\bot \leq c_0,\, 1 > c_0,\,c_0 < 1
  %   \\[.5ex]
  %   \G_4 = w_1 : \bot >  c_0,\,w_1:p \leq c_1,\,\D
  %   \\[.5ex]   
  %   \G_5 = c_2\geq 1,\,w_1:\bot \geq c_2,\, 1 > c_0,\,  w_1:p \leq c_1,\,\D
  %   \\[.5ex]    
  %   \G_6 =  w_1 : p > w_1 :\bot,\, w_1 :\bot\leq c_2 ,\,w_1:\bot \geq c_2,\, 1 > c_0,\,  w_1:p \leq c_1,\,\D
  % \end{array}   
  % \]
  \caption{$\Calcw$-derivation of $w:...$.}
  \label{fig:der}
\end{figure}

\begin{lemma}\label{lemma:soundRule}
  Let $\rho$ be an instance of a rule of the calculus $\Calcw$, let
  $\Gamma$ be the conclusion of $\rho$ and let $\M$ be a \GWM-model.
  If $\M\models\G$, then there exists a premise $\G'$ of $\rho$ such
  that $\M\models\G'$.
\end{lemma}

\begin{proof}
  Let $\rho$ be the rule $\ruleAxat$; since the rule has no premises,
  we have to show that there is no $\M$ such that $\M\models \G$.  Let
  us assume, by contradiction, that there is a \GWM-model $\M=\stru{W,R,e}$ and an
  $\M$-interpretation $\Ical$ such that $\M\models_\Ical \G$.  If $ w
  : \Box \a > t \in \G$, then $e(\Ical(w), \Box \a) > \Ical(t)$, hence
  $1 > \Ical(t)$, which implies $\M\models_\Ical 1 > t$.  Similarly,
  if $ w : \Diam \a < t \in \G$, then $\M\models_\Ical 0 < t$.  This
  proves that $\M\models_\Ical \Atmp{\G}$.  By exploiting $R$ and  $e$, 
 one can define a solution to $\Atmp{\G}$\footnote{Note that irrational values of $R$ and $e$ 
    must be approximated with rationals.};
  accordingly, $\Atmp{\G}$ is consistent, a contradiction.  We
  conclude that $\M$ does not exist. %
  The cases concerning the other rules are detailed in the
  Appendix~\ref{app:proofs}.
\end{proof}

\begin{proposition}[Soundness]\label{prop:sound}
  \begin{enumerate}[label=(\roman*), ref=(\roman*),leftmargin=*]    
  \item\label{prop:sound:1} If $\provesw{\G}$, then $\M\not\models
    \G$, for every \GWM-model $\M$.
    
  \item\label{prop:sound:2} $\provesw w:\varphi < 1$ implies
    $\varphi\in\GWL$.
  \end{enumerate}
\end{proposition}

\begin{proof} 
  Point~\ref{prop:sound:1} can be  proved by induction on the
  depth of a $\Calcw$-derivation of $\G$, by exploiting
  Lemma~\ref{lemma:soundRule}.  Let
  us  assume  $\varphi\not\in\GWL$.  Then, there exists a \GWM-model
  $\M=\stru{W,R,e}$ and a world $ w^\star\in W$ such that
  $e(w^\star,\varphi) < 1$.  Let $\Ical$ be an $\M$-interpretation
  mapping $w$ to $w^\star$; it holds that $\M\models_\Ical w:\varphi < 1$. 
  By~\ref{prop:sound:1},
we get $\nprovesw w:\varphi < 1$, and this
proves~\ref{prop:sound:2}. 
\end{proof}





\noindent
We show that $\Calcw$ is \emph{strongly terminating}, namely: there
exists a well-founded relation $\prec$ such that, for every
application $\rho$ of a rule of $\Calcw$, if $\G$ is the conclusion of
$\rho$ and $\G'$ any of the premises of $\rho$, then $\G'\prec \G$.
Equivalently, given a finite multiset $\G$ and repeatedly applying the
rules of $\Calcw$ upwards, proof search eventually halts, no matter
which strategy is used.


The size of a formula $\varphi$, denoted by  $\size{\varphi}$, is defined as follows
\[
\size{\varphi} \;=\;
    \begin{cases}
      0 & \mbox{if $\varphi\in\PV\cup\{\bot\}$}
          \\
\size{\a} + \size{\b} + 1  & \mbox{if $\varphi=\a\land\b$ or $\varphi=\a\lor\b$ or $\varphi=\a\to\b$}
  \\
\size{\a} + 2 & \mbox{if $\varphi=\Box\a$ or $\varphi=\Diam\a$}
    \end{cases}
\]
We extend $\wgname$ as follows ($\G$ is a  multiset of constraints):
\[\small
  \begin{array}{l}
   \wg{u} \;=\;
    \begin{cases}
      0 & \mbox{if $u$ is an atomic c-term}
          \\
      \size{\varphi} &     \mbox{if $u = w:\varphi$}
\\
 \size{\varphi} + 1  &\mbox{if $u = R(w,w')\land w':\varphi$ or $u = R(w,w')\to w':\varphi$ }
      \end{cases}
\\[6ex]
  \wg{u \abstractorder t}  \,=\, \wg{u}
  \hspace{5em}
    \wg{\G}\,=\,\sum_{\chi\in\Gamma} \wg{\chi}
\end{array}    
\]
The multisets  $\G[w]$ (multiset of constraints) and   $\sizem{\G}$ (multiset of natural numbers) are defined as follows:
\[
  \G[w] \,=\,\{\,
u  \abstractorder t\in\G~|~\mbox{$\lab{u} = w$}
  \,\}
\qquad
\sizem{\G} \,=\, \{\, \wg{\G[w]}~|~\mbox{$w$ occurs in $\G$}    \,\}
    \]
Let $\Theta_1$ and $\Theta_2$ be finite multisets of natural numbers;
we set:% \\[.5ex]
\[
  \Theta_1 \precm \Theta_2
%\qquad\IFF\qquad
\quad\mbox{iff}\quad
  \Theta_1\neq
 \Theta_2 \;\land\; \left(\; \forall k_1 \in \Theta_1\setminus
   \Theta_2.\ \exists k_2\in \Theta_2 \setminus \Theta_1.\ k_1 < k_2
  \;\right).
\]  
The relation $\precm$ is a multiset order, hence $\precm$ is
well-founded (see~\cite{BaaderN:98}, Th.~2.5.5 and Lemma 2.5.6).
We introduce the following well-founded relation $\precc$ between finite multisets
of constraints:
\[
\G_1 \precc \G_2 \quad\mbox{iff}\quad \sizem{\G_1} \precm\sizem{\G_2}. 
\]  

We can prove (see App.~\ref{app:proofs}):

\begin{lemma}\label{lemma:rulesDec}
  Let $\rho$ be an application of a rule of the calculus $\Calcw$, let
  $\Gamma$ be the conclusion of $\rho$ and  $\G'$ any of the premises of $\rho$.
  Then,  $\G'\precc \G$.
\end{lemma}

Since $\precc$ is well-founded, from Lemma~\ref{lemma:rulesDec}  the strong termination of $\Calcw$ follows:

\begin{proposition}\label{prop:term}
    The calculus $\Calcw$ is strongly terminating.
\end{proposition}


Accordingly, any backward proof search
strategy for $\Calcw$ terminates.
Following~\cite{FerFioFio:2013,FioFer:2021jlc}, we focus on strategies
where failure is certified by countermodels, i.e.: if proof search
for a multiset of constraints $\G$ fails, a \GWM-model $\M$ such that
$\M\models \G$ can be built; we call $\M$ a countermodel for $\G$,
since $\M$ ascertains that $\G$ is not provable in $\Calcw$ (by
its soundness).
Let $\G$ be a multiset of constraints:

\begin{itemize}
\item $\G$ is \emph{reduced} iff
 no rule of $\Calcw$ can be backward applied to $\G$;
 thus,  every non-atomic constraint in $\G$ has the form
  $w:\Box \a \gt t$ or  $w:\Diam \a \lt t$.

\item $\G$ is \emph{plain} iff 
  only a modal rule can  be backward applied to $\G$;
  thus,  every non-atomic constraint in $\G$ has the form
   $w:\Box \a \abstractorder t$ or  $w:\Diam \a \abstractorder t$.
 \end{itemize}
 An application of a modal rule is plain iff the conclusion is plain.
By $\app{\chi}{\G_k}{\G_{k+1}}$ we mean that $\G_{k+1}$ is a premise
of an application of a rule $\rho$ of   of $\Calcw$ having conclusion
$\G_k$ and main constraint $\chi$, with $\chi\in\G_k$. % (note that $\rho$ is determined by $\chi$).
A  branch $\Bcal$ is a sequence $\stru{\G_0,\dots,\G_n}$ such that, for every $0 \leq k < n$,
there is $\chi_k$ such that $\app{\chi_k}{\G_k}{\G_{k+1}}$; thus,
$\Bcal$ represents a branch of a $\Calcw$-tree.
The branch $\Bcal$ is \emph{saturated}
iff the following holds:
  
\begin{itemize}[leftmargin=*]
\item $\G_0$ is a finite set of constraints of the form $w:\varphi\abstractorder t$;

\item  $\G_n$ is reduced;
\item every application of a modal rule in $\Bcal$ is plain.
%  for every $k\geq 0$, if $\G_k$ is the conclusion of a modal rule, then $\G_k$ is p-reduced.
\end{itemize}

\noindent
We show that, by exploiting  $\G_n$, we can build a countermodel for $\G_0$.
Let $\Gat$ be  a finite consistent set of atomic constraints and let $\s$ be a solution to $\Gat$.
By  $\Mod{\Gat,\sigma}$ we denote the $\GM$-model  $\M=\stru{W,R,e}$
such that $W$ is the set of labels occurring in $\Gat$ and:
\[\small
  \begin{array}{l}
R(w,w') \;=\;
\begin{cases}
  \s(R(w,w')) & \mbox{if $R(w,w')$ occurs in $\Gat$}  
  \\
  0 & \mbox{otherwise}
\end{cases}
\qquad
  e(w,p) \;=\;
\begin{cases}
  \s(w:p) & \mbox{if $w:p$ occurs in $\Gat$}  
  \\
  0 & \mbox{otherwise}
\end{cases}
  \end{array}
  \]
We remark that  $\M$ is a discrete $\GWM$-model.
An interpretation  $\Ical$ of $\Mod{\Gat,\sigma}$ 
is \emph{canonical} iff
$\Ical(w)=w$ for every label $w$ occurring in $\Gat$.
% Let $\Bcal=\stru{\G_0,\dots,\G_n}$ be a saturated branch, let  $\s$ be a solution to
% $\Atmp{\G_n}$ and  $\M=\Mod{\Atmp{\G_n},\s}$;
% in Lemma~\ref{lemma:count} we show that
%  $\M\models_\Ical \G_0$, where $\Ical$ is any canonical interpretation.




\begin{example}\label{ex:satBrancj}
  Let $\varphi$ be the formula $\Box(p\lor q) \to (\Box p\lor \Diam q)$.
  The $\Calcw$-tree displayed below corresponds to  a saturated branch $\Bcal=\stru{\G_0,\dots,\G_7}$, 
  where $\G_0=\{w_0:\varphi < 1\}$ is the root and $\G_7$ the top multiset.
When the applied rule has
  two premises, the annotation $(l)$ (left) or $(r)$ (right) specifies
  the selected one; in every rule application,
  the main constraint is underlined.
\[\small
  \begin{array}{l}
% Gamma_7
    \AXC{$R(w_0,w_1)> w_1:p,\;\; w_1:p\leq c, \;\;  w_1 : q \geq R(w_0,w_1),\;\; 1> c,\;\;  R(w_0,w_1)\leq c,\;\;\D$}     
% ----------------------------------------------
% Gamma_6
    \RightLabel{$R\land\lt(l)$}
    \UIC{$R(w_0,w_1)> w_1:p,\;\; w_1:p\leq c, \;\;  w_1 : q \geq R(w_0,w_1),\;\; 1> c,\;\; \underline{R(w_0,w_1)\land w_1 : q \leq c},\;\;\D$}
% ----------------------------------------------
% Gamma_5
    \RightLabel{$\lor\gt(r)$}
 \UIC{$R(w_0,w_1)> w_1:p,\;\; w_1:p\leq c, \;\;  \underline{w_1 :p\lor q \geq R(w_0,w_1)},\;\; 1> c,\;\; R(w_0,w_1)\land w_1 : q \leq c,\;\;\D$}
% -------------------------------------------------------------
% Gamma_4
    \RightLabel{$R\to\gt(l)$}
\UIC{$R(w_0,w_1)> w_1:p,\;\; w_1:p\leq c, \;\;  \underline{R(w_0,w_1)\to w_1 :p\lor q > c},\;\; R(w_0,w_1)\land w_1 : q \leq c,\;\;\D$} 
% ----------------------------------------------
% Gamma_3
    \RightLabel{$R\to\leq(r)$}
    \UIC{$\underline{R(w_0,w_1)\to w_1 :p \leq c},\;\;  R(w_0,w_1)\to w_1 :p\lor q > c,\;\; R(w_0,w_1)\land w_1 : q \leq c,\;\;\D$} 
  % ----------------------------------------------
%Gamma_2
    \RightLabel{$\Box\lt$}
 \UIC{$w_0:  \Box(p\lor q) > c,\;\; \underline{w_0 : \Box p \leq  c},\;\; w_0: \Diam q  \leq  c,\;\;  c   < 1$} 
%----------------------------------------------
% Gamma_1
    \RightLabel{$\lor\lt$}
\UIC{$w_0:  \Box(p\lor q) > c,\;\;\underline{w_0 : (\Box p\lor \Diam q) \leq c},\;\; c  < 1$}
%----------------------------------------------
% Gamma_0
    \RightLabel{$\to <$}
\UIC{\underline{$w_0:  \Box(p\lor q) \to (\Box p\lor \Diam q)  < 1$}}
    \DP
\\[15ex]
% ###############################################
c = w_0:q_0\qquad\D\;=\;\{\,     w_0:  \Box(p\lor q) > c,\;\;  w_0: \Diam q  \leq  c,\;\;  c   < 1 \,\}    
  \end{array}
    \]  
 The set $\Atmp{\G_7}$ consists of the following atomic constraints:
    \[\small
  R(w_0,w_1)> w_1:p,\;\; w_1:p\leq c, \;\;  w_1 : q \geq R(w_0,w_1),\; 1> c,\;\;  R(w_0,w_1)\leq c,\;\; c < 1  
    \]
The set   $\Atmp{\G_7}$ is consistent; a solution to $\Atmp{\G_7}$ is any substitution $\s$ over  $\Atmp{\G_7}$ such that
    \[
      \sigma(w_1:p)\, < \,\sigma\left(R(w_0,w_1)\right)\,\leq\, \sigma(c) \,< 1
      \qquad
      \sigma\left(R(w_0,w_1)\right)\,\leq \, \sigma(w_1: q) %\,\leq\, 1
\]
For every solution $\s$, the \GWM-model $\M=\Mod{\Atmp{\G_7},\s}$ satisfies
$\M \models_\Ical w_0 :\varphi < 1$, where $\Ical$ is a canonical interpretation,
hence $\M$ is a countermodel for $\varphi$, witnessing that $\varphi\not\in\GWL$.
A  solution $\s^\star$ is  obtained by setting $\sigma^\star(w_1:p) = 0.5$ and
$\sigma^\star(R(w_0,w_1))= \sigma^\star(c) =  \sigma^\star(w_1:q) = 0.6$.
The model $\Mod{\Atmp{\G_7},\s^\star}$  is essentially the same as the one displayed in Fig.~\ref{fig:countGW},
the only   difference is in the evaluation of $q_0$ in $w_0$.
\EndEx
\end{example}



\begin{lemma}\label{lemma:count}
  Let $\Bcal=\stru{\G_0,\dots,\G_n}$ be a saturated branch and  $\s$  a solution to
  $\Atmp{\G_n}$.
  Let $\M=\Mod{\Atmp{\G_n},\s}$,   $\Ical$   a  canonical interpretation
  of $\M$ 
  and    $\chi\in \bigcup_{k\in\{0,\dots,n\}}\G_k$.
Then, $\M\models_\Ical \chi$.
\end{lemma}


\begin{proof}
  Since $\M=\Mod{\Atmp{\G_n},\s}$, with $\s$ is a solution to $\Atmp{\G_n}$,
  it holds that:
\begin{enumerate}[label=(\arabic*), ref=(\arabic*),leftmargin=*]
% (Atmp)
 \item\label{lemma:count:atmp}
  $\M\models_\Ical \Atmp{\G_n}$.
\end{enumerate}
We prove  $\M\models_\Ical \chi$ by induction on $\wg{\chi}$;
we assume $\M=\stru{W,R,e}$.
 Let $\wg{\chi}=0$; then $\chi$ is atomic, hence $\chi\in\G_n$,
  and $\M\models_\Ical \chi$ by~\ref{lemma:count:atmp}.
  
  Let $\wg{\chi}>0$ and let us assume that  $\chi\not\in \G_n$.
  Then, there exists $k \geq  0$ such that
  $\app{\chi}{\G_k}{\G_{k+1}}$. 
We proceed by a case analysis on the structure of $\chi$.


  %%%%%%%%%%%%%%%% AND GT
  Let $\chi= w:\a\land \b\gt t$. 
  Since  $\app{\chi}{\G_k}{\G_{k+1}}$,
the applied rule is $\land\gt$, hence 
  $w:\a\gt t\in \G_{k+1}$
  and $w:\b \gt t\in \G_{k+1}$. 
  By the induction hypothesis   $\M\models_\Ical w:\a\gt t$ and $\M\models_\Ical w:\b \gt t $, and 
this implies  $\M\models_\Ical w:\a\land \b\gt t$.
The case $\chi= w:\a\lor \b\lt t$  is similar.

%%%%%%%%%%%%%%%%  AND LT
Let $\chi= w:\a\land \b\lt t$.
Since  $\app{\chi}{\G_k}{\G_{k+1}}$, the applied rule is $\land\lt$, hence
$w:\a\lt t \in \G_{k+1}$ or $w:\b\lt t  \in \G_{k+1}$.
By the induction hypothesis $\M\models_\Ical  w:\a\lt t $ or $\M\models_\Ical w:\b\lt t $, hence
 $\M\models_\Ical  w:\a\land \b\lt t$.
 The case  $\chi= w:\a\lor \b\gt t$ is similar.
 
 
 %%%%%%%%%%%%%%%%  IMP LTS  
Let $\chi= w:\a\to \b < t$; 
we only consider the case  $\b\not\in\PV\cup\{\bot\}$.
Since  $\app{\chi}{\G_k}{\G_{k+1}}$,
the applied rule is $\to<$, hence:
 \begin{enumerate}[label=(\Alph*), ref=(\Alph*),leftmargin=*]
% (A)
 \item\label{lemma:count:impless:1}
 $\{\,w:\a > w : q,\, w:\b \leq w : q,\,w: q < t\,\} \subseteq \G_{k+1}$. 
\end{enumerate}
 By the induction hypothesis,
for every constraint $\chi'$  in~\ref{lemma:count:impless:1},
  $\M\models_\Ical \chi'$, hence:
\[
e(w,\a) > e(w,q) \qquad e(w,\b) \leq e(w,q) \qquad e(w,q) < \Ical(t)
\]  
Since $e(w,\a) > e(w,\b)$, we get  $e(w,\a\to \b) = e(w,\b)$, hence  $e(w,\a\to \b) < \Ical(t)$,
which implies $\M\models_\Ical  w:\a\to \b < t  $.


%%%%%%%%%%%%%%%% IMP LTE
Let $\chi= w:\a\to \b \leq t$; 
we only consider the case  $\b\not\in\PV\cup\{\bot\}$.
Since   $\app{\chi}{\G_k}{\G_{k+1}}$,
the applied rule is $\to\leq$,
hence one of the followings two subcases holds:

 \begin{enumerate}[label=(B\arabic*), ref=(B\arabic*),leftmargin=*]
% (B1)
 \item\label{lemma:count:impleq:1}
    $t\geq 1\in \G_{k+1}$;

% (B2)
    \item\label{lemma:count:impleq:2}
$\{\,w:\a> w :q,\,w:\b\leq w:q,\,w:q\leq t\,\} \subseteq \G_{k+1}$.
\end{enumerate}
Let us assume that~\ref{lemma:count:impleq:1} holds.
By the induction hypothesis, we get $\M\models_\Ical t\geq 1$,
and this implies
$\M\models_\Ical w:\a\to \b \leq t$.
Let us assume that~\ref{lemma:count:impleq:2} holds.
By the induction hypothesis, for every constraint $\chi'$  in~\ref{lemma:count:impleq:2},
 $\M\models_\Ical \chi'$, hence:
\[
e(w,\a) > e(w,q) \qquad e(w,\b) \leq e(w,q) \qquad e(w,q) \leq \Ical(t)
\]  
Since $e(w,\a) > e(w,\b)$,
we get  $e(w,\a\to \b) = e(w,\b)$, hence  $e(w,\a\to \b) \leq \Ical(t)$,
which implies $\M\models_\Ical  w:\a\to \b \leq t$.


%%%%%%%%%%%%%%%% IMP GT
Let $\chi= w:\a\to \b \gt t$;
we only consider the case  $\b\not\in\PV\cup\{\bot\}$.
Since   $\app{\chi}{\G_k}{\G_{k+1}}$,
the applied rule is $\to\gt$, hence
one of the followings two subcases holds:

 \begin{enumerate}[label=(C\arabic*), ref=(C\arabic*),leftmargin=*]
% (C1)
 \item\label{lemma:count:implgt:1}
$\{\,w:\a \leq w:q,\,w:\b\geq  w:q,\,1\gt t\,\}\subseteq\G_{k+1}$;



% (C2)    
    \item\label{lemma:count:implgt:2}
$w:\b\gt t\in\G_{k+1}$.
\end{enumerate}

\noindent
Let us assume that case~\ref{lemma:count:implgt:1} holds.
By the induction hypothesis,   for every constraint $\chi'$  in~\ref{lemma:count:implgt:1},
 $\M\models_\Ical \chi'$, hence:
\[
e(w,\a) \leq  e(w,q) \qquad e(w,\b) \geq e(w,q) \qquad 1 \gt \Ical(t)
\]  
Since $e(w,\a)\leq  e(w,\b)$, we get $e(w,\a\to \b)=1$, hence  $e(w,\a\to \b)\gt \Ical(t)$,
which implies $\M\models_\Ical\chi$.
In case~\ref{lemma:count:implgt:2}, by the induction hypothesis we get $\M\models_\Ical w:\b\gt t$,
namely $e(w,\b)\gt\Ical(t)$.
Since $e(w,\a\to\b)\geq e(w,\b)$, we get $e(w,\a\to\b)\gt\Ical(t)$, hence
$\M\models_\Ical   w:\a\to \b \gt t$.
The cases concerning $R$-constrains can be proved as the corresponding propositional cases.


%%%%%%%%%%%%%%%% BOX LT

Let $\chi= w:\Box\a\lt t$.
Since  $\app{\chi}{\G_k}{\G_{k+1}}$, the applied rule is $\Box\lt$, hence:
 \begin{enumerate}[label=(\Alph*), ref=(\Alph*),leftmargin=*,start=4]
   
 % (D)
 \item\label{lemma:count:boxlt}
%$\chi_1\in \G_{k+1}$, where $\chi_1\,=\,R(w,w_1)\to w_1:\a\lt t$.
$R(w,w_1)\to w_1:\a\lt t \, \in \G_{k+1}$.
 \end{enumerate}
Let $r_1$ be the value of $R(w,w_1)\to e(w_1:\a)$.
By the induction hypothesis
  $\M\models_\Ical R(w,w_1)\to w_1:\a\lt t $, hence $r_1\lt\ \Ical(t)$.
Since  $e(w,\Box\a)\leq r_1$, we get
  $e(w,\Box\a) \lt \Ical(t)$, and this implies
  $\M\models_\Ical  w:\Box\a\lt t$.
The case  $\chi= w:\Diam\a\gt t$ is similar.


%%%%%%%%%%%%%%%%%%%%%%%  BOX GT 

It remains to consider the case where  $\wg{\chi}>0$ and   $\chi\in \G_n$,
thus $\chi= w:\Box\a\gt t$ or  $\chi= w:\Diam\a\lt t$.
Let $\chi= w:\Box\a\gt t$;
we show that:

 \begin{enumerate}[label=(\Alph*), ref=(\Alph*),leftmargin=*,start=5]
% (E)
 \item\label{lemma:count:boxgt}
$\M\models_\Ical R(w,w') \to w':\a\gt t$, for every  $w'\in W$.
\end{enumerate}
Let $w'\in W$; 
we show that  $\M\models_\Ical R(w,w') \to w':\a\gt t$.
Let us assume that  $R(w,w')$ does not occur in $\G_n$.
By definition of $\M$, the value of $R(w,w')$ is 0, hence 
 we have to show  that $\M\models_\Ical 1 \gt t$.
The case where $\gt$ is $\geq$ is immediate. 
Let $\gt$ be  $>$.
Since  $\chi\in\G_n$, we get $1 >t\in\Atmp{\G_n}$,
thus $\M\models_\Ical1 >  t$ by~\ref{lemma:count:atmp}.
Let us assume that  $R(w,w')$  occurs in $\G_n$.
Then, there exists $k > 0$  be such that $R(w,w')$ is introduced in  $\G_{k}$ by an application of a modal rule $\rho$, namely:
\begin{itemize}
\item $\app{\chi'}{\G_{k-1}}{\G_k}$, with $\chi' = w :\Box \varphi \lt t'$ or  $\chi' = w :\Diam\varphi \gt t'$,
  and the new label introduced by $\rho$ is  $w'$.
\end{itemize}
Since $\Bcal$ is saturated,   $\G_{k-1}$ is plain,
hence the non-atomic constraints in   $\G_{k-1}$ are modal.
As a consequence, 
no rule applied in the sub-branch $\stru{\G_{k-1},\dots,\G_n}$
can introduce in the premise  new constraints  $u\abstractorder t'$ such that
$u$ has label $w$;
this implies  that   $\chi\in \G_{k-1}$.
By definition of $\rho$,
we get $\Phibd{\G_{k-1},w,w'}\subseteq \G_{k}$, hence
$R(w,w') \to w':\a\gt t \in\G_{k}$. By the induction hypothesis,
we get  $\M\models_\Ical R(w,w') \to w':\a\gt t$, and
this concludes the proof of~\ref{lemma:count:boxgt}.
Let $e(w,\Box\a)=r$.
By definition, there exists $w^\star\in W$ such that
the value of  $R(w,w^\star) \to e(w^\star,\a)$ is $r$.
By~\ref{lemma:count:boxgt}, $r\gt \Ical(t)$, 
namely  $e(w,\Box\a)\gt \Ical(t)$;
 we conclude  $\M\models_\Ical w:\Box\a\gt t$.
The case $\chi= w:\Diam\a\lt t$ is similar.
\end{proof}



\begin{proposition}\label{prop:satBranch}
  Let $\Bcal=\stru{\G_0,\dots,\G_n}$ be a saturated branch.

  \begin{enumerate}[label=(\roman*), ref=(\roman*),leftmargin=*]
  \item\label{prop:satBranch:1}
Let $\M=\Mod{\Atmp{\G_n},\s}$, with  $\s$  a solution to
  $\Atmp{\G_n}$. Then, $\M\models \G_0$.
    
  \item\label{prop:satBranch:2}
    If $\G_0=\{ w : \varphi < 1\}$, then $\varphi\not\in\GWL$.
  \end{enumerate}
\end{proposition}





Let $\Bs$ be a backward proof search strategy for $\Calcw$; we say that
$\Bs$ is \emph{plain} iff all the modal rule applications performed by
$\Bs$ are plain.


\begin{lemma}\label{lemma:search}
  Let $\Bs$ be a plain proof search strategy for $\Calcw$ and let
  $\G_0$ be a finite multiset of constraints of the form $w:\varphi\abstractorder t$.  If $\Bs$ fails to prove
  $\G_0$, then a discrete countermodel for $\G_0$ can be built.
\end{lemma}


\begin{proof}
  Assume that   $\Bs$ fails.
  By tracking the computation, we can build an open branch
  $\Bcal$  having root $\G_0$. 
  Since $\Bs$ is plain, the branch $\Bcal$ is saturated,
  hence from $\Bcal $ a discrete countermodel for $\G_0$ can be extracted
 (see Prop~\ref{prop:satBranch}\ref{prop:satBranch:1}).
\end{proof}

\noindent
We remark that $\Bs$ does  not need to implement backtracking.



\begin{proposition}[Completeness]\label{prop:compl}
  \begin{enumerate}[label=(\roman*), ref=(\roman*),leftmargin=*]
  \item\label{prop:compl:1} If $\nprovesw{w: \varphi < 1}$, then there exists a
discrete   for countermodel for $\varphi$.
  \item\label{prop:compl:2}
    If $\nprovesw{w : \varphi < 1}$, then  $\varphi\not\in\GWL$.
  \item\label{prop:compl:3} If $\varphi\not\in\GWL$, then then there
    exists a discrete countermodel for $\varphi$.
  \end{enumerate}
\end{proposition}

\begin{proof}
  Point~\ref{prop:compl:1} follows from Lemma~\ref{lemma:search}.
  Point~\ref{prop:compl:2} follows from~\ref{prop:compl:1},
  since a countermodel for $w :\varphi < 1$ certifies that $\varphi\not\in\GWL$.
  Point~\ref{prop:compl:3} follows from~\ref{prop:compl:1} since, by soundness of $\Calcw$,
  $\varphi\not\in\GWL$ implies $\nprovesw{w : \varphi < 1}$.
\end{proof}

\noindent
\inlinetodo{**OLD**}

By Prop.~\ref{prop:compl}\ref{prop:compl:3}, $\GWL$ has the finite
model property. Let $\varphi\not\in\GWL$ and let $\M$ be the discrete
countermodel extracted from an open branch having root
$w_0:\varphi<1$.  One can easily prove that the depth of $\M$ is
bounded by $\size{\varphi}$ and that every world of $\M$ has at most
$\size{\varphi}$ R-successors; this implies that the size of $\M$ is
$O(\size{\varphi}^{\size{\varphi}})$ (see
Th.~4.1~\cite{BilkovaFK:22}).  Note that it is not possible to build
$\M$ one branch at a time, since the constraints generated during the
expansion of a branch have global validity and must be kept throughout
the construction.  However, by adapting the procedure described in
Th.~4.2~\cite{BilkovaFK:22}, one can prove that the decision problem
for $\GWL$ is in PSPACE.


We have implemented both the proof search procedure and the
countermodel extraction in the JTabWb
framework~\cite{FerrariFF:17a}. The prover, named
\texttt{gwcref}~\cite{gwcrefProver}, performs a standard backward
depth-first proof search, leveraging the JTabWb engine complemented
with the Java implementation of the $\Calcw$ rules and of a plain
proof-search strategy. The consistency of atomic constraints is
checked using the Choco-solver Java library~\cite{ChocoSolver:2025}.
To our knowledge, the only other prover available for modal fuzzy
logic is mNiBLoS~\cite{Vidal:16}, an SMT-based solver designed for
continuous t-norm-based logics, including G\"odel-Dummett
Logic. However, mNiBLoS does not support the logic $\GWCL$.

% Our prover, called \texttt{gwcref}
% (see~\cite{JtabwbProvers}), utilizes the Choco-solver Java
% library~\cite{ChocoSolver:2025} to verify the consistency of atomic
% constraints.  Additionally, it provides functionality for generating
% \LaTeX~representations of both derivations and countermodels.  To our
% knowledge, the only other prover available for modal fuzzy logic is
% mNiBLoS~\cite{Vidal:16}, an SMT-based solver designed for continuous
% t-norm-based logics, including G\"odel-Dummett Logic. However, mNiBLoS
% does not support the logic $\GWCL$.


  





%%% Local Variables: 
%%% mode: latex
%%% TeX-master: "goedelModalLogicWitnessNonCrisp"
%%% End: 

%\input{04-kripkesem.tex}
%\input{05-future.tex}



% \subsubsection*{Acknowledgments}
% \begin{credits}
%   % We thank the reviewers for their valuable and constructive comments.
%   This project has received funding from the European Union's Horizon
%   2020 research and innovation programme under the
%   Marie Sk{\l}odowska-Curie grant agreement No 101007627.
%   Camillo Fiorentini is member of the Gruppo Nazionale Calcolo Scientifico-Istituto
%   Nazionale di Alta Matematica (GNCS-INdAM).
% \end{credits}


%\bibliographystyle{splncs04}
\bibliographystyle{eptcs}
\bibliography{goedelModalLogicWitnessNonCrisp}

\newpage
\appendix

\section{Appendix}


\subsection{Non  interdefinability of the modal operators}\label{app:noninterdBD}

The non interdefinability of the modal operators in $\GCL$ has been proved in~\cite{RodriguezV:21} by exploiting  algebraic semantics.
Here we rephrase the argumentation using  witnessed crisp models ($\GWCM$-models);
more specifically, we show that:

\begin{enumerate}[label=(P\arabic*), ref=(P\arabic*),leftmargin=*]    
 \item\label{defBD:P1}
   there is no  $\Box$-free formula $\varphi$ such that
   $\Box p \tto\varphi$ is valid in  $\GWCL$; 

\item\label{defBD:P2}
  there is no  $\Diam$-free formula $\psi$ such that
   $\Diam p \tto\psi$ is valid in  $\GWCL$. 

\end{enumerate}
As a consequence, the modal  operators are not  interdefinable in any  modal logic contained in  $\GWCL$.
We recall that in a  $\GWCM$-model $\stru{W,R,e}$,   $R\subseteq W\times W$.


% \noindent
% Below we refer to the model  in Fig.~\ref{fig:ind}; as usual,
% in defining the evaluation relation $e$,
% we only display the values $e(w,q)$ such that $e(w,q)>0$.






\begin{lemma}\label{lemma:defBD}
  Let $\M^\ast=\stru{W,R,e}$ be the  $\GWCM$-model defined in Fig.~\ref{fig:ind}.

  
  \begin{enumerate}[label=(\roman*), ref=(\roman*),leftmargin=*]    
 \item\label{lemma:defBD:1}
   For every $\Box$-free formula $\varphi$,
   $e(w_0,\varphi)\in\{0,\,0.5,\,1\}$.

\item\label{lemma:defBD:2}
   For every $\Diam$-free formula $\psi$,
   $e(w_0,\psi)\in\{0,\,0.4,\,1\}$.

   
\end{enumerate}
   
\end{lemma}

\begin{proof}
  One can easily prove that, for every formula $\a$, the following facts hold:

  
  
  \begin{enumerate}[label=(\arabic*), ref=(\arabic*),leftmargin=*]    
 \item\label{lemma:defBD:F1}
 $e(w_1,\a)\in\{0,\,0.4,\,1\}$.

\item\label{lemma:defBD:F2}
 $e(w_2,\a)=\Phi(e(w_1,\a))$,
  where $\Phi(0)=0$,  $\Phi(0.4)=0.5$, $\Phi(1)=1$.

\end{enumerate}
We prove~\ref{lemma:defBD:1}, by induction on the structure of $\varphi$.
If $\varphi\in\PV\cup\{\bot\}$, we have $e(w_0,\varphi) = 0$.
The cases $\varphi=\a\land \b$,  $\varphi=\a\lor \b$ and  $\varphi=\a\to \b$
easily follow by the induction hypothesis.
Let $\varphi = \Diam \a$.
By~\ref{lemma:defBD:F1} and~\ref{lemma:defBD:F2},
one of the following properties~\ref{lemma:defBD:A}--\ref{lemma:defBD:C} holds:


  \begin{enumerate}[label=(\alph*), ref=(\alph*),leftmargin=*]    
 \item\label{lemma:defBD:A}
$e(w_1,\a) = e(w_2,\a) = 0$.

\item\label{lemma:defBD:B}
$e(w_1,\a) =  0.4$ and   $e(w_2,\a) = 0.5$.


\item\label{lemma:defBD:C}
$e(w_1,\a) = e(w_2,\a) = 1$.
\end{enumerate}
In case~\ref{lemma:defBD:A} we get  $e(w_0,\Diam \a)=0$, in case~\ref{lemma:defBD:B}  $e(w_0,\Diam \a)=0.5$, in case~\ref{lemma:defBD:C}
$e(w_0,\Diam \a)=1$; this concludes the proof of~\ref{lemma:defBD:1}.
The proof of~\ref{lemma:defBD:2} is similar. 

  
\end{proof}

\begin{proposition}\label{prop:gwclBD}
Properties~\ref{defBD:P1} and~\ref{defBD:P2} holds.
  \end{proposition}  

  \begin{proof}
    Let us assume, by contradiction, that property~\ref{defBD:P1} does not hold.
    Then, there exists  a  $\Box$-free formula $\varphi$ such that
    $\Box p \tto\varphi$ is valid in  $\GWCL$. As a consequence,
    in the \GWCM-model $\M^\ast$ in  Fig.~\ref{fig:ind} we have
    $e(w_0, \Box p \tto\varphi)=1$. Since    $e(w_0, \Box p)= 0.4$,
    we get $e(w_0, \varphi)= 0.4$, in contradiction with Lemma~\ref{lemma:defBD}\ref{lemma:defBD:1}.
This proves that~\ref{defBD:P1} holds. The proof of~\ref{defBD:P2} is similar.  
  \end{proof}


\begin{proposition}\label{prop:defBD}
The modal operators $\Box$ and $\Diam$ are not interdefinable in any modal logic contained in $\GWCL$.
  \end{proposition}  

  \begin{proof}
    Let $L$ be any modal logic contained in  $\GWCL$ and  let us assume
    that  in $L$ the operator $\Box$ can be defined.
    Then, there exists a $\Box$-free formula $\varphi$ such that
    $\Box p\tto\varphi\in L$. Since $L\subseteq \GWCL$, we get
     $\Box p\tto\varphi\in \GWCL$, in contradiction with Lemma~\ref{prop:gwclBD}.
The proof for  $\Diam$ is similar.
  \end{proof}


\begin{figure}[h]
  \centering
  \begin{minipage}{0.45\linewidth}
    \begin{tikzpicture}[scale=0.8]
      % WORLDS  
      % w0
      \draw[fill] (0,-0.5) circle (3pt)
      +(0,-0)   node (w0)  {}  % used to label the circle 
      +(0,-0.3) node  {$w_0$}
      ;
      
      
      % w1
      \draw[fill] (-1,1) circle (3pt)
      +(0,0)   node (w1)  {}  % used to label the circle 
      +(-0.6,0) node{$w_1$}
      +(0,0.5) node{$p=0.4$}
      ;
      
      % w2
      \draw[fill] (1,1) circle (3pt)
      +(0,0)   node (w2)  {}  % used to label the circle 
      +(-0.6,0) node{$w_2$}
      +(0,0.5) node{$p=0.5$}
      ;
      

      
      % ARROWS
      \draw[->] (w0) -- (w1);
      \draw[->] (w0) -- (w2);
    \end{tikzpicture}
  \end{minipage}
  \begin{minipage}{0.5\linewidth}
    \[\small
    \begin{array}{l}
      % -----------------------------------------      
      \fcolorbox{black}{gray!10}{
        \begin{minipage}{16em}\small
          $W=\{\,w_0,\,w_1,\,w_2\,\}$
          \\
          $R =\{\,(w_0,w_1),\,(w_0,w_2)\,\}$
          \\
          $e(w_1,p) = 0.4,\,e(w_2,p) = 0.5$  
        \end{minipage}
      } % end box    
      % -----------------------------------------  
    \end{array}
    \]
  \end{minipage}
  \caption{The \GWCM-model  $\M^\ast=\stru{W,R,e}$.}
  \label{fig:ind}
\end{figure}
  

\subsection{Proofs}\label{app:proofs}



\begin{lemmaref}{\ref{lemma:soundRule}}
  Let $\rho$ be an application of a rule of the calculus $\Calcw$, let
  $\Gamma$ be the conclusion of $\rho$ and let $\M$ be a \GWM-model.
  If $\M\models\G$, then there exists a premise $\G'$ of $\rho$ such
  that $\M\models\G'$.
\end{lemmaref}

\begin{proof}
\inlinetodo{** OLD PROOF **} 

\smallskip
Let $\G$ be the conclusion of rule $\rho$ of $\Calcw$ and let us assume
$\M\models_\Ical\G$, where $\M=\stru{W,R,e}$; we show that there
exists a premise $\G'$ of $\rho$ and an $\M$-interpretation $\Ical'$
such that $\M\models_{\Ical'}\G'$.

%If $\rho$ is an axiom rule, the assertion trivially holds since there
%is no $\M$-interpretation $\Ical$ such that $\M\models_\Ical\G$.  
The case $\rho=\ruleAxat$ has already been discussed in the paper.
If $\rho$ is one of the rules $\land\lt$, $\land\gt$, $\lor\lt$,
$\lor\gt$, the assertion easily follows.

Let us consider the application
\[
%% IMP LTS 
    \AXC{$\overbrace{w :\a > w : q,\,  w :\b \leq w :q,\,  w : q  < t,\,  \G_0}^{\G_1}$}    
    % ------------------------------------------    
    \RightLabel{$\to < \;(\dag)$}
    \UIC{$\underbrace{w: \a\to \b < t,\,\G_0}_{\G}$}
    \DP
\]
By the hypothesis, it holds that $\M\models_{\Ical} w: \a\to \b < t$
and $\M\models_{\Ical} \G_0$.  This implies that $e(\Ical(w), \a\to
\b) < \Ical(t) \leq 1$, hence $e(\Ical(w),\a) > e(\Ical(w),\b)$ and
$e(\Ical(w),\a\to \b) = e(\Ical(w),\b)$, thus $\M\models_\Ical w:\b <
t$. Let us remark that $\b\not\in \Vcal\cup{\bot}$ and that $q$ is a fresh propositional variable.  Let $e(\Ical(w),\b)\leq e(\Ical'(w),q)$ and let, for
every constraint $\chi$ not containing $q$, $\M\models_\Ical \chi$ iff
$\M\models_{\Ical'} \chi$.  Since $\M\models_{\Ical} \G_0$ and $b$
does not occur in $\G_0$, we get $\M\models_{\Ical'} \G_0$.  It is
easy to check that $\M\models_{\Ical'} w :\b \leq w:q$ and
$\M\models_{\Ical'} w :\a > w:q$ and $\M\models_{\Ical'} w:q < t$, hence
$\M\models_{\Ical'} \G_1$.

Let us consider the application
\[
%% IMP LTE 
\AXC{$\overbrace{t \geq 1,\, \G_0}^{\G_1}$}    
\AXC{$\overbrace{w :\b \leq w:q,\,  w :\a > w:q,\,  w:q \leq  t,\,\G_0}^{\G_2}$}    
% ------------------------------------------    
\RightLabel{$\to\leq$}
\BIC{$\underbrace{w: \a\to \b \leq t,\,\G_0}_{\G}$}
\DP
\]
By the hypothesis, $\M\models_{\Ical} w: \a\to \b \leq t$ and
$\M\models_{\Ical} \G_0$, hence $e(\Ical(w), \a\to \b)\leq \Ical(t)$.
Let us assume $e(\Ical(w),\a) \leq e(\Ical(w),\b)$. Then,
$e(\Ical(w),\a\to \b) = 1$, which implies $1\leq \Ical(t)$, namely
$\Ical(t) =1$.  It follows that $\M\models_\Ical t \geq 1$, hence
$\M\models_\Ical \G_1$. Let us assume $e(\Ical(w),\a) >
e(\Ical(w),\b)$.  We have $e(\Ical(w),\a\to \b) = e(\Ical(w),\b)$,
hence $\M\models_\Ical w:\b \leq t$.  We remark that
$\b\not\in\PV\cup\{\bot\}$ and that $q$ is a fresh propositional variable. Let $e(\Ical(w),\b)\leq e(\Ical'(w),q)$ and let, for
every constraint $\chi$ not containing $q$, $\M\models_\Ical \chi$ iff
$\M\models_{\Ical'} \chi$.
Since $\M\models_{\Ical} \G_0$ and $q$ does not occur in $\G_0$, we
get $\M\models_{\Ical'} \G_0$.  Moreover, one can easily check that
$\M\models_{\Ical'} w :\b \leq w:q$ and $\M\models_{\Ical'} w :\a > w:q$
and $\M\models_{\Ical'} w:q \leq t$; we conclude $\M\models_{\Ical'}
\G_2$.


Let us consider the application
\[
%% IMP GT 
\AXC{$\overbrace{w :\a \leq w:q,\,  w :\b \geq w:q,\,1\gt t,\, \G_0}^{\G_1}$}    
\AXC{$\overbrace{w:\b \gt t,\, \G_0}^{\G_2}$}
% ------------------------------------------       
\RightLabel{$\to\gt$}
\BIC{$\underbrace{w: \a\to \b \gt t,\,\G_0}_{\G}$}
\DP
\]
By the hypothesis, $\M\models_{\Ical} w: \a\to \b \gt t$ and
$\M\models_{\Ical} \G_0$, hence $e(\Ical(w),\a\to\b) \gt\Ical(t)$.  We
can reason as in the cases concerning rules $\to<$ and $\to\leq$.  If
$e(\Ical(w),\a) > e(\Ical(w),\b)$, then $e(\Ical(w),\a\to \b) =
e(\Ical(w),\b)$, hence $\M\models_\Ical \G_2$.  Let us assume
$e(\Ical(w),\a) \leq e(\Ical(w),\b)$.  In this case $e(\Ical(w),\a\to
\b) = 1$, which implies $1 \gt \Ical(t)$.  We can prove that
$\M\models_{\Ical'} \G_1$, where $e(\Ical(w), \a)\leq e(\Ical'(w),q)$.

Let us consider the application
\[
\AXC{$\overbrace{R(w,w_1)\to w_1:\a \lt t,\,\Phibd{\G,w,w_1},\,\G_0}^{\G_1}$}
% -----------------------------------------
\RightLabel{$\Box\lt$}
\UIC{$\underbrace{w:\Box\a \lt t,\,\G_0}_{\G}$}    
\DP
\]
By the hypothesis, $\M\models_{\Ical} w:\Box\a \lt t$ and
$\M\models_{\Ical} \G_0$, hence $e(\Ical(w),\Box \a)\lt \Ical(t)$.
Let us assume that the world $\Ical(w)$ of $\M$ has no $R$-successors, then $\Ical(R(w,w')=0$ and thus $\Ical(R(w,w_1)\to w_1:\a)=1$. Hence we get $1\lt t$ and $\M\models_{\Ical} \G_1$.
Let us assume that $\Ical(w)$ has at least one $R$-successor. 
Since $\M$ is a $\GWM$-model, there exists a world
$w^\star$ of $\M$ such that $R(\Ical(w),w^\star)\to e(w^\star,\a)=e(\Ical(w),\Box\a)$.  Let $\Ical'=\reass{\Ical}{w_1}{w^\star}$.  We
remark that, for every constraint $\chi$ not containing $w_1$,
$\M\models_\Ical \chi$ iff $\M\models_{\Ical'} \chi$. Since
$\M\models_{\Ical}\G_0$ and $w_1$ does not occur in $\G_0$, we get
$\M\models_{\Ical'}\G_0$. By the fact that $e(\Ical(w),\Box \a)\lt
\Ical(t)$ and $e(\Ical(w),\Box \a) = R(\Ical(w),w^\star)\to e(w^\star,\a)$, we have $R(\Ical(w),w^\star)\to e(w^\star,\a)\lt
\Ical(t)$, and this implies $\M\models_{\Ical'} R(w,w')\to w':\a\lt t$. Let
$\chi\in\Phibd{\G,w,w_1}$ and assume that $\chi=R(w,w')\to w_1 : \b \gt t$. 

We
have $w : \Box \b \gt t'\in \G_0$, hence $e(\Ical(w),\Box \b) \gt
\Ical(t')$.  Since $\Ical(w) R w^\star$, it follows that $e(w^\star,
\b) \gt \Ical(t')$, and this implies $\M\models_{\Ical'} \chi$.
Similarly, if $\chi=w_1 : \b \lt t'$, namely $w : \Diam \b \lt t'\in
\G_0$, we get $\M\models_{\Ical'} \chi$.  We conclude
$\M\models_{\Ical'} \G_2$.

The case where $\rho$ is  rule $\Diam\gt$ is similar.
\qed  
\end{proof}

  
\begin{lemmaref}{\ref{lemma:rulesDec}}
  Let $\rho$ be an instance of a rule of the calculus $\Calcw$, let
  $\Gamma$ be the conclusion of $\rho$ and  $\G'$ any of the premises of $\rho$.
  Then,  $\G'\precc \G$.
\end{lemmaref}



\begin{proof}
We only discuss two representative cases.
Let us consider the following application of rule $\to < $:
\[
  \begin{array}{l}
   \AXC{$\overbrace{w :\a > w : q,\,  w :\b \leq w :q,\,  w : q  < t,\,\D_w,\,\Theta}^{\G'}$}    
    % ------------------------------------------    
    \RightLabel{$\to < $}
    \UIC{$\underbrace{w: \a\to \b < t,\,\D_w,\,\Theta}_\G$}
    \DP
  \\[2ex]
\mbox{all the constraints in $\Delta_w$ have label $w$, no constraint in $\Theta$ has label $w$}
\\[1ex]
 \G[w] =\{w:\a\to \b < t\} \cup \D_w 
\hspace{3em}
 \G'[w] =\{\,  w :\a > w : q,\,  w :\b \leq w :q,\,  w : q  < t  \,\}\cup\D_w
  \end{array}
\]
We have ($\cup$ denotes the multiset union):
\[
  \sizem{\G} \,=\, \{\, \wg{\G[w]} \,\}\cup \sizem{\Theta}
  \hspace{3em}
    \sizem{\G'} \;=\;
  \{\, \wg{\G'[w]} \,\}    \cup \sizem{\Theta}    
\]
We prove that $\sizem{\G'}\precm \sizem{\G}$. The following holds:
\[
\begin{array}{rcl}
\wg{\G[w]}&\,=\, &\wg{w:\a\to \b < t} + \wg{\D_w} \;=\; \wg{\a} +  \wg{\b} + 1 +  \wg{\D_w} 
\\[1ex]
    \wg{\G[w']}&\,=\, & \wg{w :\a > w : q} +\wg{w :\b \leq w :q} + \wg{w : q  < t } + \wg{\D_w}
\\
               &\,=\,&  \wg{\a} +  \wg{\b} + 0 +  \wg{\D_w} 
  \end{array}
\]
Since $\wg{\G[w']} <  \wg{\G[w]}$, we get
$\sizem{\G'}\precm \sizem{\G}$, hence
$\G'\precc \G$.

%-------------------------------------------------------------------------
\smallskip
Let us consider the following  application of $\Box\lt$:
\[
  \begin{array}{l}
     \AXC{$\overbrace{\D_{w_1}, \;\D_w,\;\Theta}^{\G'}$}
     % -----------------------------------------
     \RightLabel{$\Box\lt$}
     \UIC{$\underbrace{w:\Box\a \lt t,\;\Delta_w,\;\Theta}_\G$}    
    \DP
\qquad \D_{w_1}\;=\;\{\,  R(w,w_1)\to w_1:\a \lt t\,\}\cup \Phibd{\Delta_w,w,w_1}
 \\[8ex]
\mbox{all the constraints in $\Delta_w$ have label $w$, no constraint in $\Theta$ has label $w$}
\\[1ex]
    \G[w] =\{w:\Box\a \lt t\} \cup \D_w
    \hspace{4em}
    \G'[w] =\D_w
    \hspace{4em}
     \G'[w_1]=\D_{w_1}
  \end{array}
\]
We have:
\[
    \sizem{\G} \;=\; \{\, \wg{\G[w]} \,\}\cup \sizem{\Theta} 
  \hspace{4em}
    \sizem{\G'} \;=\;
  \{\, \wg{\G'[w]},\,   \wg{\G'[w_1]} \,\}    \cup \sizem{\Theta}    
\]  
We prove that $\sizem{\G'}\precm   \sizem{\G}$. To this aim,  we show that:

\begin{enumerate}[label=(\roman*), ref=(\roman*),leftmargin=*]    
 \item\label{lemma:rulesDec:box1}
$\wg{\G'[w]} < \wg{\G[w]}$.

 \item\label{lemma:rulesDec:box2}
$\wg{\G'[w_1]} < \wg{\G[w]}$.
\end{enumerate}
The proof of point~\ref{lemma:rulesDec:box1} is immediate since $\G[w] =  \{w:\Box\a \lt t\}\cup\G'[w]$.
We prove~\ref{lemma:rulesDec:box2}. We have:
\[
  \begin{array}{l}
\wg{\G[w]}\;=\; \wg{w:\Box\a \lt t } + \wg{\Delta_w} \;=\; \wg{\a} + 2 +  \wg{\Delta_w}
\\[1ex]
    \wg{\G'[w_1]}\;=\; 
   \wg{ R(w,w_1)\to w_1:\a } + \wg{ \Phibd{\Delta_w,w,w_1} }
   \;=\;  \wg{\a} + 1 + \wg{ \Phibd{\Delta_w,w,w_1}}
\end{array}
    \]
To conclude the proof of~\ref{lemma:rulesDec:box2},
we show that:

\begin{enumerate}[label=(\roman*), ref=(\roman*),leftmargin=*,start=3]    
 \item\label{lemma:rulesDec:box3}
$\wg{ \Phibd{\Delta_w,w,w_1}} <  \wg{\Delta_w}$.
\end{enumerate}
Let $\chi_1\in   \Phibd{\Delta_w,w,w_1}$. There exists $\chi\in \Delta_w$ such that one of the two
following conditions holds:
\begin{enumerate}[label=(\alph*), ref=(\alph*),leftmargin=*]    
 \item\label{lemma:rulesDec:box3a}
$\chi = w:\Box \b \gt t'$ and $\chi_1 = R(w,w_1)\to w_1 : \b \gt t'$;

 \item\label{lemma:rulesDec:box3b}
$\chi = w:\Diam \b \lt t'$ and $\chi_1 = R(w,w_1)\land w_1 : \b \lt t'$.
\end{enumerate}
In both cases, it holds that $\wg{\chi_1}= \wg{\b} +1$ and   $\wg{\chi}= \wg{\b} +2$,
hence $\wg{\chi_1} < \wg{\chi}$;  this proves~\ref{lemma:rulesDec:box3}.
From~\ref{lemma:rulesDec:box1} and~\ref{lemma:rulesDec:box2}
it follows that $\sizem{\G'}\precm   \sizem{\G}$, hence
$\G'\precc \G$.
\end{proof}








%%% Local Variables: 
%%% mode: latex
%%% TeX-master: "goedelModalLogicWitnessNonCrisp"
%%% End: 


 
\end{document}


